% !TEX TS-program = pdflatex
% !TEX encoding = UTF-8 Unicode

% This is a simple template for a LaTeX document using the "article" class.
% See "book", "report", "letter" for other types of document.

\documentclass[11pt]{article} % use larger type; default would be 10pt

\usepackage[utf8]{inputenc} % set input encoding (not needed with XeLaTeX)

%%% Examples of Article customizations
% These packages are optional, depending whether you want the features they provide.
% See the LaTeX Companion or other references for full information.

%%% PAGE DIMENSIONS
\usepackage{geometry} % to change the page dimensions
\geometry{a4paper, margin=30mm} % or letterpaper (US) or a5paper or....
% \geometry{margin=2in} % for example, change the margins to 2 inches all round
% \geometry{landscape} % set up the page for landscape
%   read geometry.pdf for detailed page layout information

\usepackage{graphicx} % support the \includegraphics command and options

% \usepackage[parfill]{parskip} % Activate to begin paragraphs with an empty line rather than an indent

%%% PACKAGES
\usepackage{booktabs} % for much better looking tables
\usepackage{array} % for better arrays (eg matrices) in maths
\usepackage{paralist} % very flexible & customisable lists (eg. enumerate/itemize, etc.)
\usepackage{verbatim} % adds environment for commenting out blocks of text & for better verbatim
\usepackage{subfig} % make it possible to include more than one captioned figure/table in a single float
\usepackage{diagbox} % diagonal box for tables
\usepackage{pdflscape}
% These packages are all incorporated in the memoir class to one degree or another...

%%% HEADERS & FOOTERS
\usepackage{fancyhdr} % This should be set AFTER setting up the page geometry
\pagestyle{fancy} % options: empty , plain , fancy
\renewcommand{\headrulewidth}{0pt} % customise the layout...
\lhead{}\chead{}\rhead{}
\lfoot{}\cfoot{\thepage}\rfoot{}

%%% SECTION TITLE APPEARANCE
\usepackage{sectsty}
\allsectionsfont{\sffamily\mdseries\upshape} % (See the fntguide.pdf for font help)
% (This matches ConTeXt defaults)

%%% ToC (table of contents) APPEARANCE
\usepackage[nottoc,notlof,notlot]{tocbibind} % Put the bibliography in the ToC
\usepackage[titles,subfigure]{tocloft} % Alter the style of the Table of Contents
\renewcommand{\cftsecfont}{\rmfamily\mdseries\upshape}
\renewcommand{\cftsecpagefont}{\rmfamily\mdseries\upshape} % No bold!
\usepackage{fancyvrb}
\usepackage{hyperref}

%%% END Article customizations

\setlength{\parindent}{0pt}

\newcolumntype{L}[1]{>{\raggedright\arraybackslash}p{#1}}

%%% The "real" document content comes below...

\title{Apollo RTCC MFD}
\author{by indy91}
%\date{} % Activate to display a given date or no date (if empty),
         % otherwise the current date is printed 


\begin{document}
\maketitle

\section{Introduction}

The Apollo RTCC MFD provides the necessary calculation tools to fly complete Apollo missions with Project Apollo - NASSP 8.0. As much as possible it tries to replicate the same calculations, inputs and display as were used by the actual flight controllers during Apollo. Originally started to calculate the Apollo 7 rendezvous maneuvers, the MFD has expanded to include many more features which during the Apollo program were provided by Mission Control (MCC) and the Real-Time Computer Complex (RTCC).\\

\newpage
\tableofcontents
\newpage

\section{Main Menu}

The main menu is dividing the MFD in the following categories:\\
\textbf{TAR:} Targeting menu. Contains the various maneuver computation pages. \\
\textbf{PAD:} Pre-Advisory Data. Shows the PADs that the Apollo crews received during a mission.\\
\textbf{UTI:} Utility. All additional calculation pages that are not for specific maneuvers.\\
\textbf{MCC:} MCC Displays. Shows the "TV Guide", a list of displays that were available in the MOCR.\\
\textbf{PLN:} Mission Plan Table. A central feature of the maneuver planning during a mission. Currently optional.\\
\textbf{CFG:} Configuration. Various settings for the MFD.\\
\textbf{UPL:} Uplink Page. All uplinks to the AGCs and LVDC can be found here.\\

\section{Targeting}

The targeting menu consists of the many maneuver calculation pages:\\
\textbf{REN:} Rendezvous menu. Contains the calculations for rendezvous maneuvers.\\
\textbf{ORB:} Orbit Adjustment. Contains the inputs and display for the General Purpose Maneuver processor.\\
\textbf{TLI:} TLI Planning. Currently under construction.\\
\textbf{MCC:} Midcourse Correction. Contains the inputs and displays for the Translunar Midcourse Correction Processor.\\
\textbf{LOI}: Lunar Orbit. Contains the inputs and displays for the Lunar Orbit Insertion processor.\\
\textbf{ENT}: Entry. Contains the inputs and displays for the Return-to-Earth processor (RTEP).\\
\textbf{DEO}: Deorbit targeting. Contains the inputs and displays for the Retrofire Planning.\\
\textbf{DES}: DOI Targeting. Contains the inputs and display for the Lunar Descent Planning Processor (LDPP).\\
\textbf{LIF}: Lunar Liftoff. Contains the inputs and display for the Lunar Launch Window Processor (LLWP).\\
\textbf{ASC}: Lunar Ascent. Contains the inputs and display for the Lunar Ascent Integrator (LAI).\\
\textbf{ABO}: Descent abort. Contains the inputs and display for the Powered Descent Abort Program (PDAP).\\

\subsection{Rendezvous}

This MFD page contains the three main calculation tools for rendezvous maneuvers, as well as a separate section for the Skylab rendezvous profile:\\

\textbf{LAM:} Lambert targeting. Contains the inputs and display for the Two-Impulse (TI) processor.\\ 
\textbf{CDH:} CDH/NSR maneuver. Contains the inputs and display for the Coelliptic Rendezvous processor (SPQ).\\ 
\textbf{DKI:} Docking initiate. Contains the inputs and display for the Docking Initiation Processor (DKI)\\ 
\textbf{SKY:} Skylab rendezvous. Replicates the rendezvous programs of the onboard software in the AGC of the Skylab missions.\\ 

\newpage
\subsubsection{Lambert Targeting (TI)}

The MFD uses advanced algorithms to efficiently solve \href{https://en.wikipedia.org/wiki/Lambert%27s_problem}{Lambert's Problem}.
Lambert's Problem can be explained as finding the velocity vector V1 that leads to an orbit between position vectors R1 and R2 in the time DT. \\

The Lambert Targeting functionality of this MFD allows multi-revolution calculations and uses a predictor-corrector algorithm to find a solution even in a non-spherical gravity field. This functionality has its limits and will not work beyond a few revolutions. In this MFD instead of a time difference DT the user can set the GET for the maneuver (T1) and the time of arrival (T2). The position vector R2 is always the position of a target vessel or an offset to a target vessel. The displayed maneuver Delta V is the difference between the calculated V1 and the velocity at T1 before a maneuver.\\


\textbf{OPT:} Calculation option. General calculation mode, NCC/NSR maneuver sequence, TPI/TPF maneuver sequence\\
\textbf{VTI:} Time tag of the state vectors used in the calclation (MPT mode only)\\
\textbf{T1:} Maneuver Time. If the maneuver is supposed to be executed with a specified elevation angle relative to the target, input any negative time.\\
\textbf{T2:} Arrival time at the (offset) target. If this time is specified as an orbital travel angle, input any negative time.\\
\textbf{N:} The number of revolutions from the maneuver (T1) to arrival (T2).\\
\textbf{SPH:} Changes the calculation mode between spherical and non-spherical (perturbed) gravity. The Perturbed mode forces a multi-axis manuever.\\
\textbf{TGT:} The input for the target vessel. Switches between all vessels in the simulation (non-MPT mode)\\
\textbf{CLC:} Calculate the burn solution.\\
\textbf{OFF:} Set the offset from the target. In the general calculation mode use e.g. "X=2.05" to set the individual parameters. In the NCC/NSR and TPI/PTF modes the inputs are phase angle and delta height.\\
\textbf{PHA:} Choose a phase angle relative to the target vessel. This will calculate the necessary offset distance in front or behind the target.\\
\textbf{AXI:} Multi-Axis maneuver as the default. An X-Axis maneuver only consists of a prograde or retrograde impulse. This can be used to achieve phasing relative to a target, without the need to also achieve a specific relative height or position offset left or right. Useful to minimize DV for simple phasing maneuvers.\\
\textbf{BCK:} Go back to the main menu.\\

\newpage

\subsubsection{Coelliptic Maneuver Processor (SPQ)}

\paragraph{Explanation}\mbox{} \\

Coelliptic orbits are two orbits that are coplanar (identical inclination and longitude of the ascending node) and confocal (identical eccentricity and argument of periapsis). To achieve such an orbit relative to a target vessel this MFD can calculate a maneuver based on Program 33 of the AGC and the maneuver to initiate the coelliptic sequence, based on Program 32 of the AGC.\\

\paragraph{Buttons}\mbox{} \\

\textbf{INI:} Go to SPQ initialization page.\\
\textbf{VEH:} Choose which of the two vehicles is the chaser and which is the target (MPT mode only).\\
\textbf{CHA:} Threshold time for the state vector of the chaser vehicle (MPT mode only).\\
\textbf{TGT:} Threshold time for the state vector of the target vehicle (MPT mode only).\\
\textbf{MOD:} Calculation mode, CSI or CDH maneuver.\\
\textbf{TIM:} Switches between fixed GET and finding the delta height of the maneuver or fixed delta height and finding the time of ignition.\\
\textbf{TIG:} The time of the maneuver in GET. If the option is used to find the CDH time based on delta height this is an initial guess.\\
\textbf{CLC:} Calculate the burn solution.\\
\textbf{BCK:} Go back to the main menu.\\

\paragraph{Init page buttons}\mbox{} \\

\textbf{DH:} Delta height at the CDH maneuver.\\
\textbf{E:} Desired elevation angle at TPI.\\
\textbf{TPI:} Desired TPI time.\\
\textbf{BCK:} Go back to SPQ calculation page.\\
\newpage

\subsubsection{Docking Initiation Processor (DKI)}

\paragraph{Explanation}\mbox{} \\

The basic function of the DKI is to compute impulsive maneuvers; the result of these maneuver is the rendezvous of the CSM and LM spacecraft. The DKI attains a coelliptic orbit by doing three maneuvers: (1) phase, (2) height, (3) a coelliptic maneuver that puts the chaser in a coelliptic orbit with the target. From this orbit a terminal phase maneuver (TPI) and a terminal velocity match maneuver (TPF) may be performed to achieve the actual rendezvous. The Skylab rendezvous sequences with an additional height maneuver is also supported.\\

The sequence of maneuvers are defined by so called maneuver lines. This terminology stems from the Gemini program. The first apogee after orbital insertion was assigned the number 1.0, the apogee one orbit later was maneuver line 2.0 and so forth. The Gemini 10 rendezvous for example had this maneuver schedule:

\begin{enumerate}
\item N=1.0: First apogee
\item N=1.5: Height maneuver (NH)
\item N=2.0: Phase maneuver (NC1)
\item N=3.0: Coelliptic maneuver (NSR)
\end{enumerate}

This results in the actual rendezvous (TPF) happening at roughly the fourth orbit after launch, in the Gemini terminology this was called M=4.\\

Terminology:\\

\begin{center}
\begin{tabular}{ p{1cm} p{10cm} }
M&The number of spacecraft apogee nearest the rendezvous point (M=4 is a fourth-apogee rendezvous)\\
N&The in-orbit maneuver line counter. Apogee point is N=X, perigee point is N=X.5, common nodal (plane change) points are N= X.25 or X.75\\
NH&Apogee height adjustment maneuver performed at perigee end of maneuver line. NH = 1.5 denotes height adjustment near first spacecraft perigee\\
NC1&Phase adjustment maneuver performed at apogee end of maneuver line to change the catch-up rate. NC1 = 2 denotes phase adjustment near second spacecraft apogee.\\
\end{tabular}
\end{center}

\paragraph{TIG and TPI Definition}\mbox{} \\

On this page the time of ignition and time of the TPI maneuver are defined.

\textbf{VEH:} Choose the active vehicle for the rendezvous. Options are CSM or LM. The MFD automatically detects if the current vehicle is a CSM or not.\\
\textbf{PRO:} For the Skylab rendezvous profile with an additional height adjustment maneuver this option can be set, otherwise the Regular DKI option should be used.\\
\textbf{MAN:} Options for the initial maneuver line. The line can be defined for an input time, chaser apoapsis or target apoapsis. Input time is useful if the time of ignition or number of orbits to time of ignition is given. Chaser apoapsis is the typical option for ground-up Earth orbit rendezvous. Should the target be in an elliptical orbit the target apoapsis option is useful to keep the maneuver line near the line-of-apsides of the targeting, which prevents large vertical Delta V components for the coelliptic maneuver.\\
\textbf{ML:} Value of the initial maneuver line. This can be chosen arbitrarily, but usually an X.0 value is assigned to a chaser apoapsis. Another example is a CSM rescue maneuver in lunar orbit, which is supposed to happen one orbit after DOI. Then the time of DOI is used for the initial maneuver line, the value entered with this button should be 1.0 and then first maneuver should be scheduled at the 2.0 maneuver line point\\
\textbf{PHA:} Flag to describe the initial phase angle. Normally set to 0 to indicate a phase angle of -180 to 180 DEG. In this case the chaser needs to catch up if it is behind the target by e.g. 10 DEG. If the chaser is ahead of the target by 10 DEG it needs to slow down, which means a higher orbit. If the flag is set to -1 in the first case the phase angle is interpreted as -350 DEG instead of 10 DEG and the chaser will slow down, go into a higher orbit, to achieve rendezvous. This is mainly useful for rendezvous sequences that take many orbits.\\

\textbf{TPD:} Definition of the terminal phase. There are three options for both TPI and TPF, so in total six combinations. The three options are input time, time from sunrise and time from sunset. Usually the time of TPI is known from an external calculation or the time from sunrise option with a negative time is used, to set the TPI maneuver in darkness at a fixed time before sunrise.\\
\textbf{TPV:} Input value associated with the terminal phase definition. Either a GET or a time in minutes from sunset or sunrise.\\

\paragraph{Init Parameters}\mbox{} \\

\textbf{DH1:} The Delta Height at the NCC maneuver point. Only required for Skylab rendezvous profile. Usually 20 NM\\
\textbf{DH2:} The Delta Height at the NSR maneuver point. A positive value indicates the chaser vehicle being below the target. Usually 10 to 20 NM.\\
\textbf{E:} Elevation angle at TPI. Typical value is 26.6 DEG for a rendezvous from below. For rendezvous from above, for example if the CSM needs to rescue the LM in lunar orbit, the value 208.3 DEG is usually used.\\

\paragraph{Example: Apollo 10 PDI Abort}\mbox{} \\

A simple example for a DKI targeted maneuver is the Apollo 10 PDI Abort. Similar in concept to the PDI+12 maneuvers of the lunar landing missions, Apollo 10 would have done an abort maneuver at perilune after DOI to return to the CSM one orbit earlier than in the normal rendezvous plan. The sequence of maneuvers would be the abort initiation 0.5 revolutions after DOI, targeted as a phasing (NC1) maneuver. CSI or NH would follow another half revolution later, and again the CDH or NSR maneuver would be half an orbit after that. If we assign 1.0 as the maneuver line to the DOI maneuver then we would have NC1 = 1.5, NH = 2.0, NSR = 2.5, M = 3. The maneuver line definition is using the time option with the DOI TIG as input.

\newpage
\subsection{General Purpose Maneuver (GPM)}

\subsubsection{Explanation}

The following explanation was taken from IBM RTCC Apollo Programming Systems, Missions Systems, General, Volume II (NTRS ID 19730062603):

The function of the General Purpose Maneuver Processor is to provide the flight controller with two main capabilities:

\begin{enumerate}
	\item To determine the effect that a specified incremental velocity applied at a given maneuver point (along a given pitch and yaw) will have on the orbit.
	\item To determine the maneuver required to obtain a specified orbit or orbital change.
\end{enumerate}

The first capability is more commonly known as a flight controller special-maneuver request and has six options for the maneuver point:

\begin{enumerate}
	\item An equatorial (nodal) crossing
	\item A specified longitude
	\item A specified time
	\item A specified height
	\item An apogee crossing
	\item A perigee crossing
\end{enumerate}

The second capability is divided into eight types with various maneuver points:

\begin{enumerate}
	\item A plane change at a certain equatorial crossing, longitude, time, or height.
	\item A circularization maneuver at a longitude or height.
	\item A maneuver at perigee to adjust apogee or vice-versa.
	\item A maneuver to adjust the height 180$^{\circ}$ around from the maneuver point at a longitude or time.
	\item A maneuver to shift the ascending node at an optimum time, longitude, time or height.
	\item A maneuver to obtain a specified apogee and perigee at an optimum time, longitude, or height.
	\item A maneuver to shift the ascending node and adjust the height 180$^{\circ}$ around from the maneuver point at a longitude or time.
\item A maneuver to shift the line of apsides to the maneuver point and obtain a specified height 180$^{\circ}$ around from the maneuver point at a time, longitude, or height.
\end{enumerate}

The output from the GPM Processor is a display containing such maneuver information as DV, pitch, yaw, maneuver time, maneuver height, etc., and such post maneuver information as apogee and perigee heights, longitude of the ascending nodes, eccentricity, etc. A table containing the elements before and after the maneuver at the impulsive time is also output so the maneuver may be transferred to the Mission Plan Table, if desired.\\

\subsubsection{Buttons}

\textbf{SET:} Make an input for the GPM processor.\\
\textbf{<<:} Move the marker down.\\  
\textbf{>>:} Move the marker up.\\
\textbf{CLC:} Calculate the maneuver.\\
\textbf{MPT:} Create finite maneuver from impulsive burn.\\
\textbf{BCK:} Go back to the main menu.\\

\subsection{TLI Planning}

To be determined.\\
\newpage

\subsection{Midcourse Correction Processor}

\subsubsection{Introduction}

During the translunar coast phase of an Apollo mission, it is necessary to have the capability to either correct dispersions in the nominal trajectory or determine an alternate flight plan which is within the capability of the spacecraft. This capability is provided by the midcourse correction processor. The processor has the ability to correct a dispersed state vector to some nominal end conditions, reoptimize the lunar landing mission, and generate a circumlunar flyby alternate mission. The computation types to obtain these requirements are:

\begin{enumerate}
	\item The x, y, z, and t target update (XYZ midcourse mode).
	\item The best adaptive path (BAP) reoptimization.
	\item The free-return lunar flyby mode.
\end{enumerate}

One or more mission options are available under each mode. The mission options, listed below, are defined by their mode, type of return, lunar parking orbit (LPO) orientation, and whether the mission is tied to a landing site.

\begin{enumerate}
	\item X, Y, Z and T target update.
	\item Free-return, fixed LPO orientation, landing site.
	\item Free-return, free LPO orientation, landing site.
	\item	Nonfree-return, fixed LPO orientation, landing site
	\item	Nonfree-return, free LPO orientation, landing site
	\item	Circumlunar free-return flyby to nominal H\textsubscript{pc} and $^{\phi}$\textsubscript{pc}
	\item	Circumlunar free-return flyby, specified H\textsubscript{pc} and nominal $^{\phi}$\textsubscript{pc}
	\item	SPS lunar flyby to specified H\textsubscript{pc} and INCL\textsubscript{fr}
	\item	Optimized RCS flyby to desired or optimal inclination of free return
\end{enumerate}

The MCC processor implemented in the RTCC MFD presents the last state of the processor, as used for Apollo 14 through the end of the program. Certain procedural differences for using the processor with the earlier missions arise from this, but all lunar Apollo missions are fully supported.

\subsubsection{Input/Output}

Inputs for all midcourse modes fall into two categories: those from the data table (also called skeleton flight plan), and those which are manually entered during the mission by the user. The data table contains variables which are needed to execute the different options. These variables may be target parameters used in the XYZ and T mode or first guesses for certain variables. The table also contains parameters which change according to the nominal mission design and launch day (e.g., the lunar landing site). Output parameters from a BAP midcourse can be used to update the data table for later midcourse calculations or the XYZ and T midcourse mode. The data table and the manual inputs are defined in table I and II.\\

Output from the MCC program are of three types: those displayed, those that are needed for executing the midcourse maneuver, and those which update the data table. Displayed parameters are shown in table III. Output for the data table is shown in table I. BAP’s are the only options that update the data table.\\

In the RTCC MFD the MCC processor consists of three display pages: computational inputs, constraints on the solutions and the MCC tradeoff display (modelled after the real display used in Mission Control Houston). On the constraints page some of the inputs are made in the MED format (Manual Entry Device), which is the same format as was used in the real RTCC. The MEDs for the midcourse processor all have codes starting with an F, e.g. F22. The MED inputs are checked for errors and certain omissions are replaced by default values.\\

\subsubsection{Computation page}

\textbf{MAN:} Maneuver option (1 to 9), see description above.\\
\textbf{VTI:} Vector time. Time tag of the state vector from the ephemeris (mission planning mode only).\\
\textbf{IG:} Impulsive time of ignition of the midcourse maneuver.\\
\textbf{COL:} Column for the solution. The Midcourse Tradeoff display can hold up to 4 different solutions from the MCC processor.\\
\textbf{CFG:} Configuration for the midcourse maneuver, options are docked or undocked. Used to calculate certain display parameters only.\\
\textbf{SFP:} Skeleton flight plan (data table) used for initial guesses and target parameters. Table 1 usually contains the preflight data, table 2 the results of a previous BAP midcourse calculation that were transferred from the midcourse tradeoff. Tables 3 to 5 will be supported in the future.\\
\textbf{MID:} Go to midcourse tradeoff display MFD page.\\
\textbf{HPC:} Pericynthion altitude for lunar flyby modes.\\
\textbf{INC:} Free return inclination for lunar flyby modes 8 and 9. Mode 9 is further divided into mode 9A (RCS optimized flyby) and 9B (SPS optimized flyby to specified free return inclination). If the input inclination is 0, then mode 9A, the fully optimized flyby, will be used. Otherwise the specified inclination is attained. By using a plus or minus sign for the inclination an ascending or descending return can be specified (travelling from south to north and north to south at reentry respectively).\\
\textbf{CON:} Go to midcourse constraints MFD page.\\
\textbf{BCK:} Go back to previous menu.\\

\subsubsection{Constraints page}

\textbf{F22:} Azimuth constraints. Input method: “F22, Minimum Azimuth,Maximum Azimuth;” Limited to -110° to -70°. Used by modes 3 and 5. Constrains the approach azimuth to the landing site at the time of landing. Special logic is used if the min and max azimuths are set the identical. In that case the lunar orbit has a fixed orientation, although without imposing the LOI/DOI geometry. This should be done for missions which used the LOI-1/LOI-2 maneuver sequence (Apollo 8,10-12). Example: F22,-90,-90;\\
\textbf{F23:} Time constraints. Input method: “F23,TLMIN,TLMAX;” Used by modes 4-5. This sets a minimum or maximum time limit for the arrival at pericynthion. Useful for missions with stricter timing requirements for arriving in lunar orbit (Apollo 14 to 17). If ommited (input: “F23;”) The constraints are zeroed and the pericynthion time is not constraint.\\
\textbf{F24:} Reentry constraints. Input method: “F24,Flight Path Angle,Reentry Range;” Used in the free return and lunar flyby modes. Inputs are the flight path angle at entry interface and the range from entry interface to landing.\\
\textbf{F29:} Pericynthion height limits. Input method: “F29,HPMIN,HPMAX;” Used in mode 9 only. Can be used to force the solution indirectly to a different splashdown longitude.\\
\textbf{LAT:} Latitude bias for modes 8 and 9. TBD\\
\textbf{INC:} Maximum inclination for the powered return (TEI). Not enforce yet.\\
\textbf{LOI:} Apolune and perilune height of the LOI (LOI-1) ellipse.\\
\textbf{DOI:} Apolune and perilune height of the DOI (LOI-2) ellipse.\\
\textbf{REV:} Input: REVS1 REVS2. Number of orbits spent in the first (LOI to DOI/LOI-2) and second (DOI/LOI-2 to landing site) lunar orbit. REVS2 is always an integer, REVS1 can contain partial orbits.\\
\textbf{ROT:} Input: SITEROT ETA. The first parameter is the true anomaly at the landing site at the time of landing. Usually PDI should happen at perilune, which will be 15° ahead of the landing site. In that case 15 should be the input. ETA is the true anomaly of LOI on the post LOI orbit. This will usually be consistent with the REVS1 parameter, which will put DOI at perilune.\\
\textbf{PC:} Revolutions before and after the lunar orbit plane change maneuver. Used to estimate the trajectory in lunar orbit. The first parameter M is the number of orbits between the lunar landing and the plane change maneuver. The parameter N is the number of orbits between the plane change and lunar ascent.\\
\textbf{BCK:} Back to midcourse calculation page.\\
\newpage
\subsubsection{Midcourse Tradeoff Display}

\begin{figure}[h]
	\centering
		\includegraphics{./ApolloRTCCMFDFiles/RTCCMFD_MCC_Modes_359.png}
	\caption{Midcourse Tradeoff Display}
	\label{fig:RTCCMFD_MCC_Modes_359}
\end{figure}

\textbf{COLUMN:} Shows up to 4 midcourse correction solutions.\\
\textbf{MODE:} Shows the mode (1-9) that was calculated.\\
\textbf{AZ MIN:} Minimum approach azimuth at the landing site.\\
\textbf{AZ MAX:} Maximum approach azimuth at the landing site.\\
\textbf{WEIGHT:} Weight at ignition in lbs.\\
\textbf{GETMCC:} Estimated time of ignition of the midcourse correction (actual, not impulsive).\\
\textbf{DV MCC:} Total DV of the midcourse correction in feet per second.\\
\textbf{YAW MCC:} Yaw angle (out-of-plane) of the maneuver.\\
\textbf{H PYCN:} Height of pericynthion resulting from the maneuver.\\
\textbf{GET LOI:} Estimated time of ignition of LOI (actual, not impulsive).\\
\textbf{DV LOI:} Total DV of LOI maneuver.\\
\textbf{AZ ACT:} Actual approach azimuth at the landing site.\\
\textbf{I FR:} Free return inclination, Earth referenced.\\
\textbf{I PR:} Powered return (TEI) inclination, Earth referenced.\\
\textbf{V EI:} Velocity at entry interface in feet per second.\\
\textbf{G EI:} Flight path angle (gamma) at entry interface.\\
\textbf{GETTEI:} GET of the TEI maneuver.\\
\textbf{DV TEI:} Total DV of the TEI maneuver.\\
\textbf{DV REM:} DV remaining after TEI (not implemented).\\
\textbf{GET LC:} GET of splashdown.\\
\textbf{LAT IP:} Latitude of splashdown (impact point).\\
\textbf{LNG IP:} Longitude of splashdown (impact point).\\
\textbf{DV PC:} Total DV of lunar orbit plane change maneuver.\\
\newpage
\subsection{Lunar Orbit Insertion (LOI) Processor}

\subsubsection{Introduction}

The LOI processor calculates the LOI-1 maneuver for an Apollo lunar mission. The maneuver is targeted based on the following assumed trajectory profile to the landing site. All plane change is accomplished with the first burn. A second burn (LOI-2 or DOI) adjusts the inplane orbital elements so that a specified orbit occurs at the landing site. It is not not always possible to meet all desired end conditions; thus various solutions are computed.

There are four solution types, each with a positive-negative solution, for a total of eight solutions. A positive solution is one whose perilune is rotated ahead (i.e., in the direction of motion); a negative solution is one whose perilune is rotated behind (i.e., opposite to the direction of motion). The four types are as follows.

\begin{itemize}
	\item Plane solutions: obtain the desired azimuth at the landing site, giving up the lunar orbit perilune if necessary, which is if the node between the incoming trajectory (approach hyperbola) and the orbit after LOI occurs at an altitude below the desired perilune, or above the desired apolune. This is the type of LOI maneuver generally used for Apollo 12 and earlier.
\item Coplanar solutions: obtain the desired lunar orbit shape (apolune and perilune) in the plane of the approach hyperbola with a pre-hyperbolic perilune impulsive point for the positive solution and a post-hyperbolic perilune impulsive point for the negative solution. This solution type therefore has no plane change.
\item Minimum Theta solutions: obtain the desired lunar orbit shape (apolune and perilune) and minimize the wedge angle between the actual and desired lunar orbit plane within an input maximum allowable DV.
\item Intersection solutions: adjusts the first lunar orbit perilune altitude to obtain a specified altitude difference (or intersection with no altitude difference) between it and the altitude on the post-DOI lunar orbit. This is the type of LOI maneuver generally used for Apollo 13 and later and has no use if the second lunar orbit maneuver (LOI-2) is a circularization maneuver.
\end{itemize}

The LOI processor implemented in the RTCC MFD is based on the one used for Apollo 14 and later. Most capabilities of earlier programs are retained, so that all lunar missions are still supported.
\newpage

\subsubsection{Inputs}

The inputs for the LOI processor are divided in initialization parameters and computation parameters.\\

\textbf{Computation Parameters}

\textbf{INI:} Got to LOI initialization page.\\
\textbf{VTI:} Time for taking the state vector from the CSM ephemeris (MPT mode only).\\
\textbf{APO:} Apolune height after LOI.\\
\textbf{PER:} Perilune height after LOI.\\
\textbf{DVP:} Maximum DV for positive Min Theta solution.\\
\textbf{DVN:} Maximum DV for negative Min Theta solution.\\
\textbf{DIS:} Got to LOI display.\\
\textbf{AMN:} Choose the minimum approach azimuth to the landing site.\\ 
\textbf{ADS:} Choose the desired approach azimuth to the landing site.\\ 
\textbf{AMX:} Choose the maximum approach azimuth to the landing site.\\
\textbf{BCK:} Back to main menu.\\

\textbf{Initialization Parameters}

\textbf{HA:} Apolune height after DOI/LOI-2.\\
\textbf{HP:} Perilune height after DOI/LOI-2.\\
\textbf{DW:} Angle of perilune from the landing site (negative if the landing site is post-perilune).\\
\textbf{R1:} Number of revolutions in the first lunar orbit (may have a fractional part).\\
\textbf{R2:} Number of revolutions in the second lunar orbit.\\
\textbf{ETA:} True anomaly of LPO-1 for transferring from the hyperbola to LPO-1.\\
\textbf{DHB:} Altitude constraint of the intersection solutions. The bias is negative if LPO-2 is to be below the LPO-1 perilune.\\
\textbf{PLA:} A flag to specify if plane or minimum Theta nodes should be used to compute intersection solutions.\\
\textbf{BCK:} Back to LOI computation page\\ 
\newpage
\subsubsection{LOI Display}

This display is based on the actual display used by the flight controllers for Apollo 14.\\

\begin{figure}[hp]
	\centering
		\includegraphics{./ApolloRTCCMFDFiles/RTCCMFDLOIDisplayApollo14Example.png}
	\caption{LOI Planning Display}
	\label{fig:RTCCMFDLOIDisplayApollo14Example}
\end{figure}

\textbf{Display Parameters}\\
\textbf{CSM STA:} Station ID of the state vector used to target the maneuvers (not implemented).\\
\textbf{GET VECTOR:} GET of state vector used to target LOI.\\
\textbf{LAT LLS:} Latitude of the landing site in degrees.\\
\textbf{LNG LSS:} Longitude of the landing site in degrees.\\
\textbf{R LLS:} Radius of the landing site in nautical miles.\\
\textbf{REVS 1:} Revolutions in LPO-1.\\
\textbf{REVS 2:} Revolutions in LPO-2.\\
\textbf{DH BIAS:} Height bias for the intersection solutions.\\
\textbf{AZ LLS:} Desired azimuth at the landing site.\\
\textbf{FLLS:} Angle of perilune from the landing site.\\
\textbf{HALOI1:} Apolune height on first lunar orbit.\\
\textbf{HPLOI1:} Perilune height on first lunar orbit.\\
\textbf{HALOI2:} Apolune height on second lunar orbit.\\
\textbf{HPLOI2:} Perilune height on second lunar orbit.\\
\textbf{DVMAX+:} Maximum allowable DV for the positive Min Theta solution.\\
\textbf{DVMAX-:} Maximum allowable DV for the positive Min Theta solution.\\
\textbf{RA-RP GT:} Tolerance for the calculation of DVLOI2 in nautical miles.\\

\textbf{CODE:} Code for the eight possible solutions.\\
\textbf{GETLOI:} Impulsive GET of LOI ignition.\\
\textbf{DVLOI1:} Total DV of LOI-1 in feet per second.\\
\textbf{DVLOI2:} Total DV of DOI/LOI-2 in feet per second.\\
\textbf{HND:} Height of the node (impulsive LOI ignition point).\\
\textbf{FND/H:} True anomaly at LOI on the approach hyperbola (pre LOI).\\
\textbf{HPC:} Height of perilune on the first lunar orbit.\\
\textbf{THETA:} Angle between the desired lunar orbit plane and the actual achieved plane.\\
\textbf{FND/E:} True anomaly at LOI on the first ellipse (post LOI).\\
\newpage

\subsection{Deorbit Targeting}
\subsubsection{Introduction}

The retrofire targeting calculates the time of ignition and the burn parameters for a maneuver to return the spacecraft back to Earth from Low Earth Orbit. This can include a separation maneuver from the S-IVB at a fixed time before the deorbit burn. Or a shaping maneuver, that shapes the orbit before reentry, and happens at a given GET.\\

In the MFD there are pages for the definition of the separation/shaping maneuver and the retrofire maneuver respectively. To select a splashdown target and estimated time of ignition the Recovery Target Selection Display has been implemented. Finally a page to specify the type of target of the deorbit burn can be used.\\

\subsubsection{Separation/Shaping Inputs}

\textbf{SHA:} Time of ignition for the shaping maneuver. If no shaping maneuver is desired this should be set to zero.\\
\textbf{SEP:} Time in minutes that the separation maneuver will occur before the deorbit burn.\\
\textbf{THR:} Engine for the sep/shaping maneuver. Options are the SPS or the SM RCS.\\
\textbf{DV:} Fixed Delta V of the sep/shaping maneuver. Enter either a DV or a burn time, but not both.\\
\textbf{DT:} Fixed burn time of the sep/shaping maneuver. Enter either a DV or a burn time, but not both.\\
\textbf{ATT:} Attitude in LVLH coordinates of the sep/shaping maneuver. The default values are for retrograde, heads up, 31.7$^{\circ}$ window line on the horizon.\\

\textbf{UDT:} Duration of the ullage burn for the sep/shaping maneuver. Only applies to SPS maneuvers.\\
\textbf{UTH:} Number of ullage thrusters (2 or 4). Only applies to SPS maneuvers.\\
\textbf{GBL:} Calculate the gimbal angles or use system parameters. Only applies to SPS maneuvers.\\

\subsubsection{Retrofire Constraints}

\textbf{ENG:} Engine for the retrofire maneuver. Options are the SPS or the SM RCS.\\
\textbf{MOD:} Choose the burn mode for the retrofire maneuver. The options are a fixed Delta V, a fixed burn time or a maneuver to reach the reentry target line (velocity vs. flight path angle at entry interface).\\
\textbf{VAL:} Desired Delta V or burn time for the retrofire maneuver. Doesn't apply to velocity-flight path angle targeting.\\
\textbf{ATT:} Attitude mode for the retrofire maneuver. Either an input LVLH attitude or the 31.7$^{\circ}$ window line on the horizon.\\
\textbf{LVH:} LVLH attitude for the retrofire maneuver.\\
\textbf{ULL:} Ullage options for the retrofire maneuver. Only applies to the SPS.\\

\textbf{GIM:} Calculate the gimbal angles or use system parameters. Only applies to SPS maneuvers.\\
\textbf{K1:} Initial bank angle for the reentry. Held from entry interface to the specified G-Level.\\
\textbf{GC:} G-Level at which the final bank angle is started to be used.\\
\textbf{K2:} Final bank angle for the reentry. Held from the specified G-Level to drogue chute opening. In the case of target a specified latitude as well as longitude the bank angle will be reversed.\\
\textbf{BCK:} Back to deorbit targeting menu.\\

\subsubsection{Target Selection}

This program computes and displays groundtrack data for any requested 40 degrees of longitude for a requested time and starting longitude.\\

\textbf{CLC:} Calculate recovery target selection display. Inputs are a threshold time and the starting longitude.\\
\textbf{PAG:} Cycle through the pages of the display.\\
\textbf{SEL:} Choose one of the sets of coordinates on the groundtrack and save the latitude, longitude and an estimated GET for the retrofire maneuver to be used for the retrofire computation.\\
\textbf{BCK:} Back to deorbit targeting menu.\\

\subsubsection{Retrofire Calculation}

\textbf{TYP:} Choose the type of calculation. The options are type 1 (no sep/shaping maneuver) or type 2 (with sep/shaping maneuver).\\
\textbf{GET:} Estimated time of ignition for the retrofire maneuver.\\
\textbf{LAT:} Desired splashdown target latitude. Set this to a large negative value to disable targeting the latitude. The reentry profile is then a fixed bank angle and not a bank, reverse bank angle.\\
\textbf{LNG:} Desired splashdown target longitude.\\
\textbf{MD:} Maximum miss distance of the splashdown target. Can be set to a larger value to improve convergence.\\

\textbf{DIG:} Go to Retrofire Digitals display.\\
\textbf{XDV:} Go to Retrofire External DV display.\\
\textbf{SEP:} Go to Retrofire Separation display.\\
\textbf{BCK:} Back to deorbit targeting menu.\\

\newpage
\subsection{Return-to-Earth Targeting}

\subsubsection{Introduction}

The Return-to-Earth targeting calculates maneuvers for returning the spacecraft back to Earth from beyond Low Earth Orbit. For LEO maneuvers see the deorbit targeting. The objective is to calculate a single maneuver for changing the trajectory to one having safe entry-interface conditions and satisfying certain other constraints. Other constraints are dependent on the type of abort requested. The safe entry interface condition is a velocity flight path angle target line at 400,000 feet. Either of two entries maybe be specified, a shallow or a steep target line. The steep target line was used for all lunar Apollo missions.\\

Dependent upon the request, the abort maneuver will be computed considering one of three types of impact areas:

\begin{enumerate}
	\item An Unspecified Area\\
	This means that a safe reentry is guaranteed but no consideration is given to the location of the impact point (it maybe be quite undesirable).
	\item A Primary Target Point (PTP)\\
	This would be defined by a pair of latitude-longitude values. Not currently implemented in NASSP.
	\item An Alternate Target Point (ATP)\\
	This is actually not a point but is as many as five connected line segments (defined by latitude-longitude pairs) extending generally in a longitudinal direction.\\
\end{enumerate}

Both the PTP and ATP targets can be defined by manual input. The Return-to-Earth Target Table Display contains up to five PTP and five ATP target names and definitions.\\

Three general types of abort maneuvers are available.

\begin{enumerate}
	\item Time Critical Unspecified Area (TCUA)\\
	This is an inplane maneuver producing the trajectory with the earliest possible reentry. The abort is characterized by
	\begin{enumerate}
		\item Consuming all fuel provided for the maneuver, or
		\item Having the maximum allowable reentry velocity, or
		\item Having the minimum allowable time from maneuver to entry to Entry interface.
	\end{enumerate}
	\item Fuel Critical Unspecified Area (FCUA)\\
	This is an inplane maneuver requiring the least fuel to obtain a safe entry interface. Not that if the pre-abort trajectory had a safe entry interface, a request for a FCUA abort having the same target line should result in a zero DV maneuver. In practice, because of convergence tolerances, a small maneuver will be computed.
	\item ATP Abort\\
	This inplane maneuver produces a trajectory that impacts ATP trace. Multiple solutions are possible. If they exist, they are discrete and differ by twenty-four hour increments in time of landing.\\
\end{enumerate}

The abort computations are separated into three distinct steps. They are (1) the trade-off, (2) the Abort Scan Table, and (3) return-to-Earth digitals. Associated with each step in the computations is a display which summarizes the results of that step. The construction of an abort solution may be viewed as combining models of the abort maneuver, the trajectory between maneuver and entry interface, and the reentry trajectory. The three are combined in a way that will satisfy the abort criteria and the solution constraints. The trade-off, Abort Scan Table, and return-to-Earth digital computations are characterized by the models used; the models progress from less precise to more precise solutions.\\

\subsubsection{Tradeoff}

Four major questions must be answered in abort planning:
\begin{enumerate}
\item How much fuel may be expended?
\item When may the maneuver be performed?
\item How soon must splashdown occur?
\item If the target is a PTP, how large a miss is acceptable?
\end{enumerate}

There is a hidden difficulty in answering these questions — the best answer to any one of the above is not independent of the answers to the remaining three. Sometimes by relaxing tho acceptable miss by a few miles, the required velocity decreases by several thousand feet per second; sometimes shifting the time of the
maneuver by a few minutes substantially reduces the time of landing or the chance of a miss.\\
The objective of this step of the computations is to provide the user with an overall picture of the abort situation for either ATP or PTP aborts. The abort situation is defined in terms of the above four parameters (three if the target is an ATP). To this end the user supplies parameter ranges within which his solutions of interest can be found. The RTE section imposes a mesh over the solution region and examines each mesh point for a solution. Those solutions existing are used to produce analog TV displays to assist the user in arriving at an optimum answer to the four questions given above.\\
As mentioned earlier, multiple solutions having different times of landing may be available for ATP and PTP aborts. As the time of abort is varied, these times
of landing tend to vary smoothly where they exist. This leads to a natural grouping of solutions into families of solutions, each having similar times of landing.
The Tradeoff Display is a multiple page (up to five) TV display; each page contains the analog information for a different family of solutions. Two display formats are used.

\begin{enumerate}
	\item The Remote Earth ATP consisting of one analog graph, characteristic DV versus Time of Abort
	\item Near-Earth ATP consisting of three analog graphs on each page:
	\begin{enumerate}
		\item Characteristic DV versus Time of Abort\\
					One curve representing the DV required for an inplane solution that impacts the ATP trace.
		\item Time of Landing versus Time of Abort\\
					One time of landing curve corresponding to the curve in item a.
		\item Latitude of Impact versus Time of Abort\\
					One curve representing the declination of impact. Although latitude is not considered a tradeoff parameter, it is helpful to know this information.
\end{enumerate}
\end{enumerate}

The abort solutions for the Tradeoff Display are constructed by combining an impulsive velocity change, a conic (analytic two-body) trajectory, and polynomial reentry
functions. The reentry functions simulate either a high-speed G\&N entry or a constant G-level entry and are valid only if entry interface occurred on one of the
two target lines. They provide reentry range, cross range, and DT as a function of target line, entry profile and entry interface velocity, inclination, azimuth,
and latitude. Solutions constructed for the Abort Scan Table or return-to-Earth digitals will be constrained to these conic entry interface conditions.\\

Currently the tradeoff display only works in Earth reference.

\paragraph{Inputs}\mbox{} \\

\textbf{MOD:} Switch between near-Earth or remote Earth format.\\
\textbf{REM:} Choose the page (out of 5) for the remote Earth solution to be displayed\\
\textbf{SIT:} Choose the tradeoff site from the target table.\\
\textbf{TV:} Choose the vector time (MPT mode only).\\
\textbf{MIN:} Choose the minimum abort time.\\
\textbf{MAX:} Choose the maximum abort time.\\

\textbf{PAG:} Cycle between the tradeoff pages.\\
\textbf{CLC:} Calculate tradeoff solution.\\
\textbf{ENT:} Choose the entry profile.\\

\subsubsection{Abort Scan Table (AST)}

The Abort Scan Table Display is essentially a digital scratch pad that may be used to compare up to seven discrete abort solutions. One solution is inserted for
each manual AST request. These computations are more precise than the tradeoff and less precise than return-to-Earth digital computations. They consist of
an impulsive velocity change, an integrated coast trajectory, and polynomial reentry functions. The workflow to generate a Maneuver PAD or target load for a maneuver always involves first an AST calculation followed by using the Return-to-Earth Digitals.\\

Three types of AST solution can be generated. Unspecified area (time or fuel critical), specific area (ATP or PTP) and lunar search (specific site or fuel critical). The first two of these types work in both Earth and Moon sphere of influence, while the lunar search logic can only be used while in lunar orbit. The calculations that are different between Earth and Moon centered state vectors is chosen internally.\\

The two nominally used modes are lunar search for TEI and the fuel critical, unspecified area (FCUA) mode for transearth midcourse corrections.\\

\paragraph{Inputs}\mbox{} \\

\textbf{TYP:} Cycle between the three AST types.\\
\textbf{SIT:} Choose the landing site or type of abort.\\
\textbf{VTI:} Choose the vector time for the abort (MPT mode only)\\
\textbf{TIM:} Choose the time of abort, or initial guess in the case of lunar search.\\
\textbf{TDV:} Choose the maximum DV for a time critical abort.\\
\textbf{TZ:} Choose the estimated landing time for PTP and ATP aborts.\\
\textbf{AST:} Go to the AST display page.\\
\textbf{ENT:} Choose the entry profile.\\
\textbf{MD:} Choose the maximum miss distance for a PTP abort.\\
\textbf{INC:} Choose the desired return inclination (lunar reference only). Using 0 as input will optimize the DV.\\
\textbf{BCK:} Back to entry targeting menu.\\

\paragraph{Display Buttons}\mbox{} \\
\textbf{DEL:} Delete one or all AST rows.\\
\textbf{CLC:} Calculate AST solution.\\
\textbf{RTE:} Go to RTE Digitals inputs page.\\
\textbf{BCK:} Go back to AST inputs page.\\

\paragraph{Display Explanation}\mbox{} \\

\begin{figure}[hp]
	\centering
		\includegraphics{./ApolloRTCCMFDFiles/Apollo11TEIASTExample.png}
	\caption{Example Abort Scan Table}
	\label{fig:Apollo11TEIASTExample}
\end{figure}

\textbf{CODE:} Code associated with the AST solution. Starts at 101 and is incremented with each new calculation.\\
\textbf{SITE:} When using PTP or ATP mode the specified landing site is shown, e.g. MPL for Mid Pacific Line.\\
\textbf{AM:} Abort Mode. First letter is E for Earth or L for Lunar centered. Second letter is S for lunar search or D for discrete time. Remaining letters show the abort type, ATP, PTP, TCUA or FCUA.\\
\textbf{GETI:} GET of the impulsive ignition.\\
\textbf{GETV:} GET of the state vector time used in the calculation.\\
\textbf{DV:} Total Delta V of the maneuver.\\
\textbf{INCL:} Earth relative inclination of the trajectory at entry interface. A for ascending and D for descending in terms of azimuth.\\
\textbf{HPC:} Height of pericynthion, only calulated in lunar reference.\\
\textbf{VEI:} Velocity at entry interface.\\
\textbf{GEI:} Flight path angle at entry interface.\\
\textbf{GETEI:} GET of entry interface.\\
\textbf{GETL:} GET of splashdown.\\
\textbf{LAT IP:} Latitude of impact.\\
\textbf{LNG IP:} Longitude of impact.\\

\subsubsection{Return-to-Earth Digitals}

The final solution produced by the RTE section — the Return-to-Earth digital solution — consists of an integrated maneuver, an integrated trajectory from the
end of the maneuver to entry interface, and an integrated reentry. Two solutions, each the result of a manual request, may be viewed with the Return-to-Earth
digital display. Either of these solutions may be transferred into the Mission Plan Table, used to initiate execution of the spacecraft setting study aid, or used to generate a command load.\\
Computation of a solution is similar to the AST computations in that the same iteration algorithm is used to adjust independent and dependent variables to meet
certain constraints on the dependent variables. The major differences can be listed:

\begin{enumerate}
	\item The first guess is a converged solution being viewed in the AST Display.
	\item The constraints at entry interface are taken from the entry interface state of the converged solution.
	\item Independent variables are the target parameters for the finite maneuver integrator (PMMRKJ).
	\item The reentry parameters are always obtained from an integrated reentry after the iteration converges to the entry interface conditions.
\end{enumerate}
The same precision trajectory logic is used by the iteration algorithm except that now the finite maneuver integrator is used to perform the maneuver instead of
making an impulsive velocity change. The user may specify to the Return-to-Earth digital computations any of four thrusters (SPS, SMRCS, DPS, or LMRCS).
In addition to requesting a solution constructed as described above, the user may manually define a solution by defining, via the manual entry device (MED) a time
for a vector fetch, a time of abort, and maneuver targets. This type of manual entry bypasses the iteration logic. The coast Encke integrator is used to propagate the fetched vector to the time of abort. The maneuver is integrated one time using the targets supplied. The coast Encke is used again to propagate the burnout vector to 400,000 feet at which point the reentry integrator is used to propagate to impact.

\paragraph{Inputs}\mbox{} \\
\textbf{COL:} Solution will be shown in either primary or manual column.\\
\textbf{AST:} Choose AST code for the calculation.\\
\textbf{REF:} Choose REFSMMAT type for the calculation.\\
\textbf{MAN:} Choose maneuver code for the calculation. The code consists of four letters. The first letter is the spacecraft performing the maneuver, C for CSM and L for LM. The second letter is the thruster used for the maneuver. S for SPS, D for DPS or R for RCS. The third letter is the spacecraft configuration. D for docked, A for ascent stage docked and U for undocked. The last letter is always X for External DV guidance.\\
\textbf{ULL:} Choose the ullage thrusters and duration for the burn. The ullage thruster options are 2 or 4. If the burn is an RCS burn then the options are +2, +4, -2 or -4 and these will be the thrusters used for the maneuver.\\
\textbf{TRM:} Choose the trim angle option (calculated or system parameter)\\

\textbf{DIS:} Go to the RTE Digitals display\\
\textbf{DOC:} Choose the docking angle during the maneuver.\\
\textbf{HEA:} Choose heads up or down for the maneuver.\\
\textbf{ITE:} Choose iterate or not iterate for the maneuver. Iterate is the more accurate solution, but takes longer to calculate and has a small risk of not converging.\\
\textbf{BCK:} Go back to entry targeting page.\\
\paragraph{Display Buttons}\mbox{} \\

\textbf{CLC:} Calculate RTE digitals solution.\\
\textbf{SPL:} Save the splashdown target from either primary or manual column.\\
\textbf{TRA:} In non-MPT mode the TIG and DV are save to be used for uplink and Maneuver PAD. In MPT mode the maneuver gets transferred to the MPT. The input is the MED format M74. To transfer the primary column enter "M74,CSM,,RTEP;" for the manual column "M74,CSM,,RTEM;"\\
\textbf{BCK:} Go back to RTE Digitals inputs page.\\
\newpage
\paragraph{Display Explanation}\mbox{} \\

\begin{figure}[hp]
	\centering
		\includegraphics{./ApolloRTCCMFDFiles/Apollo11TEIRTEDExample.png}
	\caption{Example Return-to-Earth Digitals}
	\label{fig:Apollo11TEIRTEDExample}
\end{figure}

\textbf{GETR:} Reference time in GET of an event.\\
\textbf{STA ID:} Station ID of the state vector used in the calculation (MPT mode only).\\
\textbf{AM:} Abort mode, see AST display explanation.\\
\textbf{GETV:} GET of vector, see AST.\\
\textbf{AREA:} Splashdown area, see AST.\\
\textbf{THR:} Maneuver code, see RTED inputs.\\
\textbf{MATRIX:} REFSMMAT used for the calculation.\\
\textbf{WT:} Total weight at main engine on.\\
\textbf{TAA:} True anomaly after abort.\\
\textbf{EP:} Primary entry profile used to generate entry simulation. HGN for G\&N or HB1 for constant G reentry.\\
\textbf{RLH:} Roll angle at ignition in LVLH coordinates.\\ 
\textbf{PLH:} Pitch angle at ignition in LVLH coordinates.\\ 
\textbf{YLH:} Yaw angle at ignition in LVLH coordinates.\\ 
\textbf{RO:} Roll/outer gimbal angle at ignition in IMU coordinates.\\
\textbf{PI:} Pitch/inner gimbal angle at ignition in IMU coordinates.\\
\textbf{YM:} Yaw/middle gimbal angle at ignition in IMU coordinates.\\
\textbf{VC:} Delta V to be used in the EMS DV counter for the burn.\\
\textbf{BT:} Burn time of the maneuver, main engine on to cutoff.\\
\textbf{VT:} Total Delta V of the maneuver.\\
\textbf{U:} Number and direction of RCS thrusters used for ullage or as the main engines for the maneuver.\\
\textbf{DT:} Ullage duration.\\
\textbf{PETI:} Phase elapsed time of ignition, relative to GETR.\\
\textbf{GETI:} Ground elapsed time of ignition.\\
\textbf{GMTI:} GMT (since midnight launch day) of ignition.\\
\textbf{BU:} Backup entry profile.\\
\textbf{PETIR:} Phase elapsed time of initial roll (usually 0.05g)\\
\textbf{LV:} Lift vector orientation, initial roll angle.\\
\textbf{GIR/GCON:} G level of initial roll if constant G iteration was the backup entry profile. Otherwise constant G level to be used to generate backup impact coordinates.\\
\textbf{GMAX:} Maximum G level encountered during the reentry.\\
\textbf{PETEI:} Phase elapsed time of entry interface.\\
\textbf{VEI:} Velocity at entry interface.\\
\textbf{GEI:} Flight path angle at entry interface.\\  
\textbf{LAT LNG EI:} Latitude and longitude at entry interface.\\
\textbf{LAT LNG ML2:} Latitude and longitude at splashdown if primary entry mode skipped out and maximum lift was used for second entry\\
\textbf{LAT LNG T:} Latitude and longitude of the splashdown target\\
\textbf{LAT LNG ZL2} Latitude and longitude at splashdown if primary entry mode skipped out and zero lift (ballistic) was used for second entry\\
\textbf{LAT LNG IPB:} Latitude and longitude at splashdown with the backup entry mode.\\
\textbf{GETL:} GET at drogue chute deployment using the primary entry mode:\\
\textbf{MD:} Miss distance of the primary entry mode to the target splashdown coordinates. In nautical miles.\\

\newpage
\subsection{Launch Window Processor}

The launch window processor (LWP) can be used to calculate the optimal liftoff time and launch targeting parameters for Saturn IB missions to a rendezvous target in orbit, such as Skylab and ASTP. At present there is no capability to stop and re-start the countdown for Saturn launches, so only at the predetermined time in a launch scenario can the launch be done. The liftoff time options other than input time should therefore not used for now.\\

\subsubsection{Inputs}

\textbf{LOT:} Liftoff time option. The options are: 

\begin{itemize}
\item{"Input time", lift-off on input time.}
\item{"Phase angle offset", compute lift-off time to achieve a desired phase angle (OFFSET) at insertion.}
\item{"Biased phase zero (GMTLOR)", lift-off on GMTLO* plus BIAS, using input time as threshold.}
\item{"Biased phase zero (TPLANE)", lift-off on GMTLO* + BIAS, using TPLANE as threshold time.}
\item{"In-plane", lift-off based on inplane launch time (GMTLO = TPLANE + TRANS).}
\item{"In-plane with nodal regression", iterate on lift-off on inplane launch time based on target orbit phase angle (final GMTLO = TYAW + TRANS).}
\end{itemize}

\textbf{TLO:} Input liftoff time.\\
\textbf{RINS:} Radius of insertion in meters.\\
\textbf{VINS:} Velocity of insertion in meters per second.\\
\textbf{GINS:} Flight-path angle of insertion in degrees.\\

\textbf{TGT:} Select target vehicle in orbit for rendezvous.\\
\textbf{NOF:} Flag for option to compute differential nodal regression from insertion to rendezvous.\\
\textbf{DNO:} Angle that is added to the target descending node to account for differential nodal regression.\\
\textbf{DIS:} Go to launch targeting display.\\ 
\textbf{BCK:} Go back to utilities page.\\

\newpage
\subsubsection{Display}

\begin{figure}[hp]
	\centering
		\includegraphics{./ApolloRTCCMFDFiles/LaunchTargetingDisplay.png}
	\caption{Launch Targeting Display for Skylab 2}
	\label{fig:LaunchTargetingDisplayExample1}
\end{figure}

\textbf{GMTLO:} Greenwich mean time of lift-off, in hours, minutes, seconds\\
\textbf{TINS:} Greenwich mean time of insertion, in hours, minutes, seconds\\
\textbf{GMTLO*:} Greenwich mean time of phase match, in hours, minutes, seconds\\
\textbf{PFA:} Powered flight arc.\\
\textbf{PFT:} Powered flight time.\\
\textbf{DN:} Descending node of chaser.\\
\textbf{TPLANE:} Greenwich mean time of inplane launch, in hours, minutes, seconds\\
\textbf{AZL:} Optimum launch azimuth.\\
\textbf{LATLS:} Geocentric latitude of launch site.\\
\textbf{LONGLS:} Geographic longitude of launch site.\\
\textbf{DELNO:} Angle between the target and chaser descending nodes, defined at insertion.\\
\textbf{DELNOD:} Rate of change of DELNO, defined at insertion.\\
\textbf{TYAW:} Greenwich mean time of lift-off to achieve minimum yaw steering, in hours, minutes, seconds\\
\textbf{TGRR:} Greenwich mean time of guidance reference release\\
\textbf{VIGM:} Velocity magnitude at insertion, in meters per second and feet per second.\\
\textbf{RIGM:} Radius magnitude at insertion, in meters and feet\\
\textbf{GIGM:} Flightpath angle at insertion\\
\textbf{IIGM:} Inclination at insertion\\
\textbf{TIGM:} Angle measured from launch site meridian to chaser descending node, defined at TGRR\\
\textbf{TDIGM:} Rate of change of TIGM, defined at TGRR\\
\textbf{APOGEE:} Height of apogee in nautical miles\\
\textbf{PERIGEE:} Height of perigee in nautical miles\\
\textbf{INCLINATION:} Inclination of orbit plane\\
\textbf{INS PHASE:} Phase angle at insertion\\
\textbf{DN TARGET:} Descending node of target\\
\textbf{BIAS:} Time added to GMTLO* time to obtain a lift-off time\\
\textbf{T ANOMALY:} True anomaly\\
\textbf{ALTITUDE:} Height of chaser at, insertion\\
\textbf{DH:} Delta height between chaser and target, at insertion\\
\textbf{TIME:} Greenwich mean time of lift-off\\

\newpage
\subsection{REFSMMAT}

\subsubsection{Explanation}

The REFSMMAT (REFerence to Stable Member MATrix) is a rotation matrix relating the Apollo Basic Reference Coordinate System (BRCS) and the currently used IMU Stable Member Coordinate System. Depending on the mission phase the REFSMMAT is chosen, so that the IMU angles provide meaningful attitude values. Some types of REFSMMATs can be calculated by the AGC itself, but most were uplinked to the spacecraft from the ground. The REFSMMATs that can be calculated with this MFD are:

\begin{itemize}
	\item{Launch: Calculates the Launch REFSMMAT, which is also calculated internally in the AGC at liftoff.}
	\item{Landing Site: Not used for Apollo 7 or 8}
	\item{PTC: Passive Thermal Control, not used for Apollo 7 or 8.}
	\item{LOI-2: A special LVLH REFSMMAT for Apollo 8, calculated before the last translunar Midcourse Correction.}
	\item{P30: Alignment for a thrusting maneuver, equivalent to option 1 in Program 52.}
	\item{P30 retro: Alignment for a retrograde burn, useful for Earth orbit reentry maneuvers.}
	\item{LVLH: Local Vertical alignment, equivalent to option 2 in Program 52.}
	\item{Lunar Entry: Equivalent to option 2 in Program 52 with the GET of Entry Interface.}
\end{itemize}
	
\subsubsection{Buttons}
	\textbf{TIM:} The options "Landing Site"', "PTC", "`P30", "P30 retro " and "LVLH" require a time in GET to calculate the REFSMMAT. For a Landing Site REFSMMAT the time chosen is either the predicted landing or launch time. The time for P30 and P30 retro REFSMMATs is the maneuver time and is set on a maneuver calculation page (Lambert, CDH or Entry).\\
	\textbf{TYP:} Choose betwen uplinking the REFSMMAT or the desired REFSMMAT. The desired REFSMMAT is the alignment, that Program 52 will align the platform to, based on the knowledge of the attitude referenced to the old, currently used REFSMMAT. Only in rare cases the REFSMMAT itself would be uploaded, e.g. when activating the Lunar Module or if the difference to the previous REFSMMAT is very small. In doubt, uplink the desired REFSMMAT!\\
	\textbf{DWN:} Downlink the current REFSMMAT from the AGC. If the type of REFSMMAT is known, select it by cycling through the REFSMMAT types by pressing OPT before doing the downlink. Useful for calculating PADs with a REFSMMAT not calculated by the RTCC MFD.\\
	\textbf{MCC:} The calculated REFSMMAT usually depends heavily on the current orbit. If there is a maneuver between now and the set time or the reentry time, change the setting to MCC to take the maneuver into account. The LOI-2 REFSMMAT is special, because the calculation of two maneuver is required before the LOI-2 REFSMMAT can be calculated. This will be explained in more detail on the Lunar Insertion page.\\
	\textbf{OPT:} Switch between the different options.\\
	\textbf{CLC:} Calculate the REFSMMAT.\\
	\textbf{UPL:} Uplink the REFSMMAT to the AGC.\\
	\textbf{LAT:} Only for Landing Site: Choose the latitude of the landing site.\\
	\textbf{LNG:} Only for Landing Site: Choose the longitude of the landing site.\\
	\textbf{BCK:} Go back to the main menu.\\

\subsection{State Vector}

\subsubsection{Explanation}

The state vector of the vessel can be calculated and uplinked here. Additionally to the functionality in the Project Apollo MFD, this MFD can calculate a state vector in the future, which sometimes was used during the Apollo program to prevent an internal state vector integration of the AGC.\\
The AGC has two slots for state vectors: CSM and LM. For the CSM the MFD will prevent uplinking a state vector that is not the vessel itself. The vessel for the LM can be freely chosen.\\

\subsection{Buttons}

\textbf{MOD:} Choose  between calculating the state vector "now" and at a specified GET.\\
\textbf{TIM:} Set the desired GET for the state vector in GET mode.\\
\textbf{TGT:} Set the target vessel.\\
\textbf{SLT:} Switch between the slots.\\
\textbf{CLC:} Calculate a state vector.\\
\textbf{UPL:} Uplinks the calculated data to the AGC.\\
\textbf{BCK:} Go back to the main menu.\\

\subsection{Landmark Tracking}

\subsubsection{Explanation}

On the Landmark Tracking page coordinates on the spherical bodies (Earth and Moon) in Orbiter 2010 can be converted to AGC coordinates. Also the contents of a Landmark Tracking PAD can be calculated. These are used for the correct timing of a pitchdown maneuver for better tracking with Program 22.

T1 is the time at which the CSM comes over the horizon and becomes visible from the landmark. At this time the astronaut can begin looking at the landmark to find the specific point he wants to track. \\
T2 is the time at which the CSM is at an elevation angle of 35$^{\circ}$ from the landmark. If any marks on the landmark are to be done, then at this time the pitchdown maneuver should be started. In Earth orbit this is usually 0.5$^{\circ}$/s, in lunar orbit 0.3$^{\circ}$/s. \\
The other displayed values are the distance of the landmark from the ground track of the CSM orbit and the AGC inputs. The AGC uses geodetic latitude, longitude divided by 2 and altitude in nautical miles as the inputs.\\

\subsubsection{Buttons}

\textbf{TIM:} Estimated time over the landmark.\\
\textbf{LAT:} Geocentric latitude of the landmark.  If the landmark is listed in a marker file, then that latitude should be used as an input here.\\
\textbf{LNG:} Longitude of the landmark.\\
\textbf{CLC:} Calculate AGC coordinates and Landmark Tracking PAD.\\

\subsection{Map Update}

\subsubsection{Explanation}

The Map Update is very different in Earth and Moon orbit. In Earth orbit the next ground station with the times of acqusition and loss of signal (AOS and LOS) are displayed.  In lunar orbit a few more times are displayed: loss of signal (LOS), sunrise (SR), crossing of the prime meridian (PM), acqusition of signal (AOS) and sunset (SS) are shown. These values are written down on the Apollo 8 Map Update forms.\\ 

\subsubsection{Buttons}

\textbf{CLC:} Calculate map update.\\
\textbf{MOD:} Cycle between Earth and Moon orbit.\\

\subsection{Maneuver PAD}

\subsubsection{Explanation}

The Maneuver Pre-Advisory Data (PAD) contains all necessary numbers to safely conduct a burn with the SPS or RCS. A complete explanation of each item on the PAD can be found in all Apollo flight plans, e.g. \href{http://history.nasa.gov/alsj/a11/a11fltpln_final_reformat.pdf}{here}. Additionally to the Maneuver PAD the very similar Apollo 7 TPI PAD was added as a second mode.\\ 

\subsubsection{Buttons}

\textbf{VEH:}The vehicle configuration is only displayed here and chosen on the configuration page.\\
\textbf{ENG:} Choose between the Service Propulsion System (SPS) and the Reaction Control System (RCS) for the maneuver.\\
\textbf{HEA:} Choose between conducting the maneuver in a heads-up or a heads-down orientation.\\
\textbf{TIG:} If you want to display a Maneuver PAD for a maneuver not calculated with the Apollo RTCC MFD you can manually enter the desired Time of Ignition and Delta V.\\
\textbf{DV:} See above.\\
\textbf{CLC:} Calculate the missing numbers on the Maneuver PAD.\\
\textbf{OPT:} Switch between the Maneuver PAD, the Apollo 7 TPI PAD and the TLI PAD.\\
\textbf{REQ:} Request a maneuver solution calculated with LTMFD or IMFD.\\
\textbf{BCK:} Go back to the main menu.\\

\subsection{Entry PAD}

\subsubsection{Explanation}

The Entry PAD contains all numbers to conduct a safe reentry. There are two types of Entry PADs: Earth Orbit Reentry and Lunar Entry. A complete explanation of each item on the PAD can be found in most Apollo flight plans, e.g. \href{http://history.nasa.gov/alsj/a11/a11fltpln_final_reformat.pdf}{here}. \\

\subsubsection{Buttons}

\textbf{MAN:} For a lunar entry you can choose between calculating a direct Entry PAD without any additional maneuvers or a Entry PAD for a previously calculated Midcourse Correction. For an Earth orbit entry a deorbit maneuver has to be performed in any case.\\
\textbf{DWN:} Downlink the splashdown coordinates from the AGC.\\
\textbf{CLC:} Calculate the missing numbers on the Entry PAD.\\
\textbf{OPT:} Switch between the Earth Entry PAD and the Lunar Entry PAD.\\
\textbf{BCK:} Go back to the main menu.\\

\subsection{VECPOINT}

\subsubsection{Explanation}

The VECPOINT page, named after a routine in the AGC, is calculating the IMU angles to point a specific part of the CSM or LM in the direction of a celestial object/astronomical body. Any body present in Orbiter 2010 can be chosen. The X-axis of the spacecraft is along its longitudinal axis, so +X is pointing the CSM directly at the body and the SPS engine directly away from it.\\

\subsubsection{Buttons}

\textbf{BOD:} Type the name of the body e.g. Sun, Moon etc.\\
\textbf{DIR:} Choose the direction of the spacecraft to be pointed at the celestial object.\\
\textbf{CLC:} Calculate the IMU angles.\\

\subsection{Configuration}

\textbf{MIS:} Choose the mission number or manual options. Used to load mission specific parameters that are valid for any launch day of the mission.\\
\textbf{TYP:} Choose the type of vehicle configuration (CSM or LM, docked or undocked).\\
\textbf{STA:} Choose the type of LM configuration that is currently being used, ascent or descent configuration.\\
\textbf{SXT:} Change the time of the sextant star check, which is part of the procedure for a normal maneuver. During Earth orbit missions the Earth often blocks the sextant from viewing many stars, so adjusting the time of the check before the maneuver allows the MFD to find a suitable star.\\
\textbf{UPL:} Inhibit or enable uplinks during times of no available ground stations. Currently all ground stations being used for Apollo 7 are implemented.\\
\textbf{DAT:} Choose the launch date for the mission.\\
\textbf{TIM:} Choose the launch time for the mission. This will update the launch time of the CSM stored in the RTCC.\\
\textbf{EPO:} Choose the AGC epoch. Usually this is a MJD at around January 1st of the yearly coordinate system defining period. This value should be automatically chose correctly for the AGC version in use.\\
\textbf{UPD:} Update the liftoff time automatically by downlinking that time from the CMC or LGC. This will actually update three values in the RTCC: CSM liftoff time, time of clock zeroing in the CMC, time of clock zeroing in the LGC. These times are normally all set to the actual liftoff time of the CSM to get a consistent basis to calculate Ground Elapsed Time in the RTCC.\\
\textbf{BCK:} Go back to the main menu.\\

\section{Mission Planning}

\section{Example: Apollo 7 Rendezvous}

This MFD can be used to replicate the ground solutions for the rendezvous and other SPS burns during the Apollo 7 mission. As an example the inputs for the following maneuvers will be presented:

\begin{enumerate}
	\item {Separation burn at 3:20:00 GET.}
	\item {NCC1 burn at 26:25:00 GET.}
	%\item {NCC2 burn at 27:30:00 GET.}
	\item {NSR burn at 28:00:00 GET.}
	\item {TPI burn at ca. 29:25:00 GET.}
\end{enumerate}

\subsection{Separation burn}

These calculations should be done shortly before the time of the maneuver. The following steps have to be done for the separation burn:

\begin{itemize}
	\item{Maneuver time (T1) is at 003:20:00h GET.}
	\item{The time for the next maneuver (T2) will be at 026:25:00h.}
	\item{The time between T1 and T2 is 23:05h, which can be calculated as about 15.4 revolutions with the current orbital period. The correct value for the input N is therefore 15.}
	\item{AXI: The phasing maneuver was an x-axis only maneuver, so this option should be chosen here.}
	\item{SPH: 15 orbits is too long a time to calculate the maneuver with non-spherical gravity. Therefore choose the option "Spherical".}
	\item{The target vessel of the rendezvous is the Apollo 7 SIVB, which has the name "AS-205-S4BSTG".}
	\item{At the arrival time the CSM has to be 70NM in front of the SIVB. Set this value pressing OFF and type "X=70" to set a 70NM offset in front (positive x-axis) of the S-IVB stage. }
	\item{A value for YOFF would be "Left" or "Right" from the vessel at arrival time. This is not desired, so this can be left as zero. A ZOFF value is not specified, so this should remain 0 for now.}
\end{itemize}

The resulting DV vector should be close to (-1.7,0,0). These values can now be used for P30 in the AGC or directly uplinked.\\

\subsection{NCC1 burn}

At 26:25:00 GET a SPS burn was executed that will put the CSM on a trajectory resulting in a phase angle of 1.32$^{\circ}$ behind and 8NM below the SIVB at 28:00:00GET. The required inputs are here:

\begin{itemize}
	\item{T1 is set to 26:25:00 GET (NCC1 time).}
	\item{T2 is set to 28:00:00 GET (NSR time).}
	\item{The time between T1 and T2 is with 1:35h slightly longer than an orbital period. No good results were found with N set to 0, so it should be set to 1.}
	\item{AXI: Because a precise position relative to the S-IVB is desired for the rendezvous sequence, the option multi-axis should be chose.}
	\item{TGT is  the same as before.}
	\item{For this short, 90 minute transfer between T1 and T2 the "Perturbed" calculation option can be used.}
	\item{The phase angle function can be used to create the x-offset. The value -1.32$^{\circ}$ results in approx. --82.58 NM for XOF.}
	\item{The ZOF value in the CSM coordinate system is positive for an offset below the target. 8NM is used for ZOFF during the coelliptic rendezvous phase.}
\end{itemize}

The resulting burn solution should be close to the vector (66.5, -1.8, 180.5). This can be used in a P30/P40 automatic SPS burn with the CSM.

%\subsection{NCC2 burn}

%This is essentially a midcourse correction burn to remove any deviations from the desired trajectory. All parameters used in the NCC1 stay the same, with two exceptions. The maneuver time T1 has a GET of 27:30:00. The number of revolutions from NCC2 to NSR also is now clearly 0, instead of 1.\\
%If the NCC1 burn was accurate, then this will result in a very low DV. During testing it never exceeded 2 feet per second.

\subsection{NSR burn}

The NSR burn nominally happens at 28:00:00 GET and places the CSM in a coelliptic orbit with a constant delta height to the target. On the CDH page of the MFD the inputs for the burn are the GET (028:00:00) and the Delta Height (DH) of the orbit, which is 8 NM for Apollo 7. Because the GET is variable, the option "Find GETI" should be used. A positive value here means below the target. When calculating the burn, the new time for the maneuver is also displayed below the number for DH. The new time is chosen, so that the delta height of the burn is exactly the specified 8NM. The results should be close to:

\begin{itemize}
	\item{028:00:30 GET}
	\item{DX: -92.7 fps}
	\item{DY: +1.6 fps}
	\item{DZ: -106.2 fps}
\end{itemize}

These numbers can be used for the external DV program (P30).

\subsection{TPI burn}

The TPI maneuver nominally was calculated by the AGC itself, but a backup solution was calculated on the ground. This backup solution can be replicated with the MFD.\\
On the Lambert page first set the S-IVB as the target. Then press T1 and type "E=27.45". The MFD will now try to find the T1, when an elevation angle of 27.45$^{\circ}$ occurs. To find the T2, which is 35 minutes after T1, press T2 and type "T1+35min". T2 will now be set to that time. Leave N as zero, calculation mode to "Perturbed" and the three offset coordinates to zero. Usual values for the maneuver:

\begin{itemize}
	\item{29:21:38 GET}
	\item{DX: +13.7 fps}
	\item{DY: +0.9 fps}
	\item{DZ: -7.9 fps}
\end{itemize}
	
On the Maneuver PAD page press OPT and CLC to display the TPI PAD.\\

%\section{Example: Apollo 8}

%\subsection{LOI-2 REFSMMAT}

%The REFSMMAT for the Lunar Orbit Circulation (LOI-2) REFSMMAT "is such that if a horizontal, in-plane, heads-up, posigrade burn were being made at LOI-2, the gimbal angle (FDAI) readout would be approximately 0,0,0. (R-P-Y)."\\

%To achieve these conditions calculate a preliminary LOI-1 burn with IMFD or LTMFD and receive the numbers for the burn with Project Apollo MFD. If you have the time of ignition and the Delta V components, open the Apollo RTCC MFD and set these numbers on the Maneuver PAD page via the manual TIG and DV inputs. Now go back to the main menu and select the REFSMMAT page and choose LVLH as the REFSMMAT option. The required time will be set to the TIG of the LOI-2 maneuver, which is 73:30:54 GET. Press CLC and uplink the solution. \\

\section{Example: Midcourse Correction Planning}
\subsection{Example 1: Apollo 11 MCC-2}

As mentioned in the introduction of this section, the version of the midcourse correction processor implemented in the RTCC MFD was used for Apollo 14 to 17. Apollo 11 would have used mode 2 of the processor for their MCC-2. Modes 2 and 4 were changed to only apply to the LOI/DOI maneuver sequence of those later missions. The same capability was retained in modes 3 and 5 though, by inputting the same desired azimuth as min and max azimuth on the constraints page.\\

To start off the calculation, go to the MCC page, under Maneuver Targeting, Midcourse. This takes you to the computation page of the processor. Here choose mode 3 (click twice on the MAN button). The Apollo 11 MCC-2 happened at about 26:45:00 GET, so press the TIG button and input that time. The other inputs can be left as they are. The solution will be shown in column 1, the maneuver is docked and the skeleton flight plan table no. 1 contains the preflight estimates. Press CON to check that all the constraints are as desired, especially the min and max azimuth constraints being identical. This should already be preloaded in the MFD, so no changes are necessary.\\ Back to the previous page (BCK button) and then to the midcourse tradeoff display (MID button) and everything should be ready for the calculation. Press CLC and the solution for the mode 3 calculation should be displayed in column 1. Using the Apollo 11 Before MCC-2 scenario that comes with NASSP this results in a maneuver of 19.7 ft/s (DV MCC).\\

You can now try different inputs and constraints, but if you are happy with the solution, you should now save the resulting data table from the MCC-2 calculation for use in the later MCC-3 and MCC-4 calculations. That is done by pressing the F30 button and typing: \emph{F30,1;} The result can be checked under MCC Display, MSK button, "1597" input, F31 button. The F31 cycles between the preflight (table 1) and the nominal (table 2) targets. Only table 2 will be saved in scenarios.\\

Back to the Midcourse Tradeoff page, the maneuver still has to be converted from an impulsive, instant maneuver to a finite maneuver taking the thruster being used into account. For that click on the MPT button and then on the THR button until the thruster of your choice is selected. SPS is set by default and is the correct choice for the maneuver. Click the CLC button and the actual TIG and DV have now been generated. These can be used to display e.g. a Maneuver PAD for the midcourse.\\

\subsection{Example 2: Apollo 11 MCC-4}

MCC-4 will use the nodal targets (latitude, longitude, radius and time of the desired position at LOI, if it was an instant velocity change maneuver) that resulted from the MCC-2 calculation, and were stored in skeleton flight plan table number 2. MCC-4 is a mode 1 maneuver with a time of ignition of about 70:55:00 GET. Input this value with the TIG button, then press the SFP button so that it says 2, for SFP table 2. Go to the midcourse tradeoff display and press CLC. Converting it to a finite maneuver works the same way as for MCC-2. Possibly MCC-4 will be small enough to be done with the RCS.\\

\subsection{Example 3: Apollo 12 MCC-2}

Apollo 12 was the first mission to fly the so called hybrid mission profile. After TLI the trajectory is ideally free return, but with a pericynthion altitude higher than necessary for successful lunar orbit insertion. Therefore there has to be a midcourse correction that takes the trajectory to a close encounter with the Moon, but in the process making it non-free return. This maneuver was planned for MCC-2. After MCC-2 the same LOI-1 and LOI-2 maneuvers were flown as on the previous missions.\\

To accomplish the hybrid transfer maneuver, mode 5 of the midcourse processor has to be used. Use a TIG of about 30:53h GET, column 1, docked vehicle configuration and table 1 (preflight targets) of the skeleton flight plan. Use the MID button to go to the midcourse tradeoff display and then press CLC to calculate the solution. It should now have calculated fairly large maneuver, the nominal Delta V being 68.8 ft/s.\\

The trajectory after MCC-2 is not constrained to be free return, but was instead optimized for the smallest possible Delta V. This will also potentially have moved the time of ignition for LOI-1 away from the flight plan time. If it is not critical that the DV optimal solution for MCC-2 is being used, the LOI-1 TIG can be adjusted by constraining that time on the constraints page. This was done on the actual Apollo 12 mission and is not possible for free-return missions like Apollo 11.\\

The flight plan time for LOI-1 is 83:25:18.2 GET. Check which time ("GET LOI") the midcourse tradeoff page is showing after calculating the MCC-2 solution. If the times are different by more than a few seconds go to the midcourse constraints page and press the F23 button. The times that are input here are not directly the LOI GET, but instead the pericynthion time, which is a bit later than LOI-1. The convergence of the constrained solution works best if the minimum and maximum times input with the F23 button are 10 minutes apart. Usually the LOI TIG will now run into the lower or upper end of the constraint. As you can't predict on which end of the time window LOI will now be, this process is always one of trial and error. So start by inputting a min and max time that contains the GET of LOI. Then recalculate the solution on the midcourse tradeoff page and check what the new LOI GET is. By moving the time window to later or earlier times with the F23 button the LOI GET can be moved as well. This might take a few iterations until the LOI GET is close to the desired time. A few seconds off are acceptable\\

After this solution resulting in the desired LOI-1 time has been calculated the process is the same as for the Apollo 11 MCC-2. After committing to performing the maneuver as calculated use the F30 button to store the data table containing the numbers required to calculate MCC-3 and MCC-4. These are calculated with mode 1 just like for Apollo 11.\\

\subsection{Example 4: Apollo 13 MCC-2}

\newpage
\section{Example: Deorbit Targeting}
\subsection{Example 1: Apollo 11 Pre TLI Abort}

Should the mission be aborted at the earliest time after reaching orbit, the first deorbit opportunity is the Atlantic ocean area, just one orbit after launch.

The CSM/S-IVB separation procedure used in computing this reentry is a 5-second SM RCS burn in the retrograde horizon monitor attitude. The separation is begun 20 minutes prior to the deorbit burn and places the CSM below and behind the S-IVB at retrofire. The horizon monitor attitude mentioned here is the nominal deorbit attitude and is attained by aligning a mark on the CM window with the earth horizon.\\

From the circular 100 NM orbit the velocity vs. flight path angle targeting for entry interface cannot be used, as there would be too little time between cutoff of the retrofire maneuver and reentry. Therefore a fixed Delta V of 325 ft/s is used, which leads to a slightly shallow reentry, but enough time for the procedures before reentry.\\

To start setting up for this calculation, review the MFD page with the inputs for the separation maneuver. The page is located under Maneuver Targeting, Deorbit, Separation/Shaping Constraints. The default values for all inputs are set up for this type of separation maneuver. The LVLH pitch angle corresponds to placing the window mark on the horizon from a 100 NM orbit. So nothing has to be changed on this page.\\

Next go to the Retrofire Constraints page. Here only the retrofire mode has to be changed from "V, Gamma" to "DV". Then enter a DV of 325 ft/s with the VAL button. The other constraints should be correctly set up. SPS engine, 31.7 deg window line, 4 quads, 15 seconds ullage, CUR REFSMMAT, compute gimbal angles. For the reentry the initial bank angle of 0 deg will be held, followed at 0.2g by a bank angle of 55 deg.\\

Next the splashdown target is selected on the Target Selection Display. Go to that page, press CLC and enter "1:0:0 -70" for a threshold GET of 1h and the desired splashdown longitude of 70 deg west. The display will now show groundtrack data, starting at that longitude. The time at that longitude should be about 1:40h GET. Now press the SEL button to select the first set of data (at 70 deg west). This will automatically generate the target for the retrofire maneuver.\\

Now go to the Retrofire Maneuver page. The estimated TIG, latitude and longitude are filled in from the target selection before. The type of maneuver has to be changed to type 2, to include the separation maneuver. Miss distance can be left at 1 NM. Now go to the main output display for the calculation, the Retrofire Digitals, with the DIG button. There the maneuver can be calculated with the CLC button.\\

\subsection{Example 2: Apollo 9 nominal deorbit}

The procedure to calculate a nominal deorbit maneuver is very similar to the one described in the previous example. Instead of the fixed Delta V of 325 ft/s the V, gamma target line is used. The calculation is of type 1 (no sep/shaping maneuver), so none of the inputs on the MFD page for the sep/shaping maneuver apply. The splashdown longitude is 59.9 deg west.

\newpage
%\renewcommand{\arraystretch}{2}
\begin{landscape}
\section{Manual Entry Device (MED) Formats}

\subsection{Acronyms}

\begin{itemize}
	\item \textbf{EBCDIC}: \href{https://en.wikipedia.org/wiki/EBCDIC}{Extended Binary Coded Decimal Interchange Code} (Characters)\\
	\item \textbf{FLP}: Floating Point\\
	\item \textbf{FXP}: Fixed Point (Integer)\\
	\item \textbf{MSK}: Manual Select Keyboard (display number)\\
\end{itemize}
\newpage

\subsection{MED List}

\textbf{MED Code}: C10\\
\textbf{Purpose}: Initiate a CMC/LGC external delta-v update\\
\textbf{Example}: \texttt{C10,CMC,1,CSM;}

\begin{center}
\begin{tabular}{|c|*{6}{>{\centering\arraybackslash}m{2.1cm}|} }
 \hline
 \diagbox{\textbf{Desc.}}{\textbf{Item}} & \textbf{1} & \textbf{2} & \textbf{3} & \textbf{4} & \textbf{5} & \textbf{6} \\ 
 \hline
 \textbf{Item Name} &Vehicle Type&Maneuver Number&MPT Indicator&&&\\
 \hline
 \textbf{Input Format} &EBCDIC&FXP&EBCDIC&&& \\
 \hline
 \textbf{Input Units} &&&&&& \\
 \hline
 \textbf{Checking Option}&Exact&Min/Max(2)&Exact(3)&&&\\
 \hline
 \textbf{Missing Item Option}&Error(1)&Error&Error&&&\\
 \hline
\end{tabular}
\end{center}

\begin{tabbing}
\textbf{Notes}:\= (1) CMC, LGC\\
\> (2) 1-15\\
\> (3) CSM/LEM\\
\end{tabbing}
\newpage

\textbf{MED Code}: G00\\
\textbf{Purpose}: CSM/LM REFSMMAT locker movement\\
\textbf{Example}: \texttt{G00,LEM,LLD,CSM,CUR;}

\begin{center}
\begin{tabular}{|c|*{6}{>{\centering\arraybackslash}m{2.1cm}|} }
 \hline
 \diagbox{\textbf{Desc.}}{\textbf{Item}} & \textbf{1} & \textbf{2} & \textbf{3} & \textbf{4} & \textbf{5} & \textbf{6} \\ 
 \hline
 \textbf{Item Name} &CSM/LEM Vehicle&Matrix 1&CSM/LEM Vehicle&Matrix 2&GET&\\
 \hline
 \textbf{Input Format} &XXX&XXX&XXX&XXX&XXX:XX:XX& \\
 \hline
 \textbf{Input Units} &EBCDIC&EBCDIC&EBCDIC&EBCDIC&HH:MM:SS& \\
 \hline
 \textbf{Checking Option}&Exact&Exact&Exact&Exact&$^{\leq}$0Current Time&\\
 \hline
 \textbf{Missing Item Option}&Error&Error&Error&Error&=Current Time&\\
 \hline
\end{tabular}
\end{center}

\textbf{Notes}: For matrix 1, valid codes are CUR, PCR, TLM, OST, MED, DMT, DOK, LCV, DOD, LLA, LLD, AGS for the LEM and CUR, PCR, TLM, OST, MED, DMT, DOD, LCV for the CSM. For matrix 2, valud codes are CUR, PCR, TLM, MED and LCV for the CSM and CUR, PCR, TLM, MED, LCV, LLA, and AGS for the LEM.\\
\newpage

\textbf{MED Code}: G03\\
\textbf{Purpose}: Compute and save local vertical CSM/LM platform alignment\\
\textbf{Example}: \texttt{G03,CSM,100:00:00;}

\begin{center}
\begin{tabular}{|c|*{6}{>{\centering\arraybackslash}m{2.1cm}|} }
 \hline
 \diagbox{\textbf{Desc.}}{\textbf{Item}} & \textbf{1} & \textbf{2} & \textbf{3} & \textbf{4} & \textbf{5} & \textbf{6} \\ 
 \hline
 \textbf{Item Name} & CSM or LEM Vehicle & GET &&&&\\
 \hline
 \textbf{Input Format} &XXX&XXX:XX:XX&&&& \\
 \hline
 \textbf{Input Units} &EBCDIC&HH:MM:SS&&&& \\
 \hline
 \textbf{Checking Option}&Exact&&&&&\\
 \hline
 \textbf{Missing Item Option}&Error&Error&&&&\\
 \hline
\end{tabular}
\end{center}

\begin{tabbing}
\textbf{Notes}:
\end{tabbing}
\newpage

\textbf{MED Code}: P08\\
\textbf{Purpose}: Update pitch angle from horizon\\
\textbf{Example}: \texttt{P08,31.6;}

\begin{center}
\begin{tabular}{|c|*{6}{>{\centering\arraybackslash}m{2.1cm}|} }
 \hline
 \diagbox{\textbf{Desc.}}{\textbf{Item}} & \textbf{1} & \textbf{2} & \textbf{3} & \textbf{4} & \textbf{5} & \textbf{6} \\ 
 \hline
 \textbf{Item Name} &Pitch Angle&&&&&\\
 \hline
 \textbf{Input Format} &FLP&&&&& \\
 \hline
 \textbf{Input Units} &degrees&&&&& \\
 \hline
 \textbf{Checking Option}&None&&&&&\\
 \hline
 \textbf{Missing Item Option}&Error&&&&&\\
 \hline
\end{tabular}
\end{center}

\begin{tabbing}
\textbf{Notes}:
\end{tabbing}
\newpage

\textbf{MED Code}: P10\\
\textbf{Purpose}: Update liftoff time for specified vehicle\\
\textbf{Example}: \texttt{P10,CSM,13:32:00;}

\begin{center}
\begin{tabular}{|c|*{6}{>{\centering\arraybackslash}m{2.1cm}|} }
 \hline
 \diagbox{\textbf{Desc.}}{\textbf{Item}} & \textbf{1} & \textbf{2} & \textbf{3} & \textbf{4} & \textbf{5} & \textbf{6} \\ 
 \hline
 \textbf{Item Name} &Vehicle&GMTALO&Traj/No traj Ind.&&&\\
 \hline
 \textbf{Input Format} &EBCDIC&H:M:S(.TH)&EBCDIC&&& \\
 \hline
 \textbf{Input Units} &VEH&hours&EBCDIC&&& \\
 \hline
 \textbf{Checking Option}&Note 1&T$\geq$0.&Exact (note 4)&&&\\
 \hline
 \textbf{Missing Item Option}&Error&Error&Insert "NO TRAJ"&&&\\
 \hline
\end{tabular}
\end{center}

\begin{tabbing}
\textbf{Notes}:\= (1) Veh. must be in MHGVNM (CSM, LEM)\\
\> (2): First veh. (MGLGMT, MCGMTL), second veh. (MGGGMT, MCGMTS)\\
\> (3): Must be "TRAJ" or "NO TRAJ" ("TRAJ" allowed in NPHASE, PRELAUNCH, PRELAUNCH 2 (L.S.))\\
\end{tabbing}
\newpage

\textbf{MED Code}: P12\\
\textbf{Purpose}: Enter GMTGRR and launch azimuth for selected vehicle\\
\textbf{Example}: \texttt{P12,CSM,13:32:00,72.0;}

\begin{center}
\begin{tabular}{|c|*{6}{>{\centering\arraybackslash}m{2.1cm}|} }
 \hline
 \diagbox{\textbf{Desc.}}{\textbf{Item}} & \textbf{1} & \textbf{2} & \textbf{3} & \textbf{4} & \textbf{5} & \textbf{6} \\ 
 \hline
 \textbf{Item Name} &Vehicle&GMTGRR&Launch Azimuth&&&\\
 \hline
 \textbf{Input Format} &EBCDIC&H:M:S(.TH)&FLP&&& \\
 \hline
 \textbf{Input Units} &EBCDIC&hours&deg.&&& \\
 \hline
 \textbf{Checking Option}&Exact 1&$\geq$0&70.$\leq$A$\leq$110.&&&\\
 \hline
 \textbf{Missing Item Option}&Error&Error&Error&&&\\
 \hline
\end{tabular}
\end{center}

\begin{tabbing}
\textbf{Notes}: \= (1) CSM, IU1, IU2. IU1 and IU2 valid at all times. CSM valid only prior to GOST initialization.\\
\> (2) See GMSMED documentation (flowchart).
\end{tabbing}
\newpage

\textbf{MED Code}: P15\\
\textbf{Purpose}: Update GMTZS for specified vehicle\\
\textbf{Example}: \texttt{P15,AGC,13:32:00;}

\begin{center}
\begin{tabular}{|c|*{6}{>{\centering\arraybackslash}m{2.1cm}|} }
 \hline
 \diagbox{\textbf{Desc.}}{\textbf{Item}} & \textbf{1} & \textbf{2} & \textbf{3} & \textbf{4} & \textbf{5} & \textbf{6} \\ 
 \hline
 \textbf{Item Name} &Vehicle&GMTZS&DT from GMTZS (LGC) to AGS on&&&\\
 \hline
 \textbf{Input Format} &EBCDIC&hours&hours&&& \\
 \hline
 \textbf{Input Units} &EBCDIC&hours&hours&&& \\
 \hline
 \textbf{Checking Option}&Exact 1&$\geq$0&$\geq$0&&&\\
 \hline
 \textbf{Missing Item Option}&Error&Ignore&Ignore&&&\\
 \hline
\end{tabular}
\end{center}

\begin{tabbing}
\textbf{Notes}: \= (1) AGC, LGC, AGS\\
\> (2) See Flowchart of GMSMED for storing details.\\
\end{tabbing}
\newpage

\textbf{MED Code}: P16\\
\textbf{Purpose}: Generate an ephemeris for one vehicle using a vector from the other vehicle\\
\textbf{Example}: \texttt{P16,CSM,LEM;}

\begin{center}
\begin{tabular}{|c|*{6}{>{\centering\arraybackslash}m{2.1cm}|} }
 \hline
 \diagbox{\textbf{Desc.}}{\textbf{Item}} & \textbf{1} & \textbf{2} & \textbf{3} & \textbf{4} & \textbf{5} & \textbf{6} \\ 
 \hline
 \textbf{Item Name} &Old vehicle&New vehicle&GMT&Maneuver number&&\\
 \hline
 \textbf{Input Format} &EBCDIC&EBCDIC&h:m:s(.th)&FXP&& \\
 \hline
 \textbf{Input Units} &EBCDIC&EBCDIC&hours&FXP&& \\
 \hline
 \textbf{Checking Option}&Exact&Exact&$\geq$0&$\geq$0&&\\
 \hline
 \textbf{Missing Item Option}&Error&Error&Insert zero&Insert zero&&\\
 \hline
\end{tabular}
\end{center}

\begin{tabbing}
\textbf{Notes}: \= (1) Vehicle must be CSM, LEM\\
\> (2) GMT and maneuver are mutually exclusive (i.e., must be one but not both).\\
\end{tabbing}
\newpage

\textbf{MED Code}: P51\\
\textbf{Purpose}: Offsets and elevation angle for two-impulse solution\\
\textbf{Example}: \texttt{P51,15,-4,26.6,130;}

\begin{center}
\begin{tabular}{|c|*{6}{>{\centering\arraybackslash}m{2.1cm}|} }
 \hline
 \diagbox{\textbf{Desc.}}{\textbf{Item}} & \textbf{1} & \textbf{2} & \textbf{3} & \textbf{4} & \textbf{5} & \textbf{6} \\ 
 \hline
 \textbf{Item Name} & Delta Height & Phase Angle &Elevation Angle of Target&Travel Angle for Terminal Phase&&\\
 \hline
 \textbf{Input Format} & FLP & FLP&FLP&FLP&& \\
 \hline
 \textbf{Input Units} &NM&deg.&deg.&deg.&& \\
 \hline
 \textbf{Checking Option}&None&None&None&None&&\\
 \hline
 \textbf{Missing Item Option}&Ignore&Ignore&Ignore&Ignore&&\\
 \hline
\end{tabular}
\end{center}

\begin{tabbing}
\textbf{Notes}:
\end{tabbing}
\newpage

\textbf{MED Code}: P52\\
\textbf{Purpose}: Two-impulse corrective combination nominals\\
\textbf{Example}: \texttt{P52,28:00:00,8,-1.32;}

\begin{center}
\begin{tabular}{|c|*{6}{>{\centering\arraybackslash}m{2.1cm}|} }
 \hline
 \diagbox{\textbf{Desc.}}{\textbf{Item}} & \textbf{1} & \textbf{2} & \textbf{3} & \textbf{4} & \textbf{5} & \textbf{6} \\ 
 \hline
 \textbf{Item Name} &Nom. Time of NSR maneuver&Nom. Height Difference at NSR&Nom. Phase Angle at NSR&&&\\
 \hline
 \textbf{Input Format} &H:M:S(.TH)&FLP&FLP&&& \\
 \hline
 \textbf{Input Units} &hours&NM&deg.&&& \\
 \hline
 \textbf{Checking Option}&$^{\geq}$0&None&None&&&\\
 \hline
 \textbf{Missing Item Option}&Error&Error&Error&&&\\
 \hline
\end{tabular}
\end{center}

\begin{tabbing}
\textbf{Notes}:
\end{tabbing}
\newpage

\textbf{MED Code}: P80\\
\textbf{Purpose}: Initialize number of vehicles, first launch vehicle, mission date\\
\textbf{Example}: \texttt{P80,1,CSM,7,16,1969;}

\begin{center}
\begin{tabular}{|c|*{6}{>{\centering\arraybackslash}m{2.1cm}|} }
 \hline
 \diagbox{\textbf{Desc.}}{\textbf{Item}} & \textbf{1} & \textbf{2} & \textbf{3} & \textbf{4} & \textbf{5} & \textbf{6} \\ 
 \hline
 \textbf{Item Name} &Number of vehicles&First Launch Vehicle&Month&Day&Year&Delta Day\\
 \hline
 \textbf{Input Format} &FXP&EBCDIC&EBCDIC&FXP&FXP&FXP \\
 \hline
 \textbf{Input Units} &NA&EBCDIC&EBCDIC&NA&NA&NA \\
 \hline
 \textbf{Checking Option}&N = 1&Exact 1&Exact 2&1$\leq$D$\leq$31&50$\leq$Y$\leq$1980&$\geq$0\\
 \hline
 \textbf{Missing Item Option}&Note 4&Note 5&Note 4&Note 4&Note 4&Note 4\\
 \hline
\end{tabular}
\end{center}

\begin{tabbing}
\textbf{Notes}: \= (1) \= Vehicle must be in MHGVNM.\\
\> (2) Date is checked by internal caldenar on final logic. Then reference day is calculated (Jan 1 =\\
\> \> day 0) along with days in month, etc. These values are stored in GZGENCSN.\\
\> (3) On entry, link to EMLAMPNP to rotate P and N matrices and sun/moon tables.\\
\> (4) If in Nophase, items 1-5 required, 6 may be input (zero inserted if missing). If in \\
\> \> any phase, items 1-5 must be missing, and item 6 must be input.\\
\end{tabbing}
\newpage

\textbf{MED Code}: U00\\
\textbf{Purpose}: Space digitals initialization\\
\textbf{Example}: \texttt{U00,CSM;}

\begin{center}
\begin{tabular}{|c|*{6}{>{\centering\arraybackslash}m{2.1cm}|} }
 \hline
 \diagbox{\textbf{Desc.}}{\textbf{Item}} & \textbf{1} & \textbf{2} & \textbf{3} & \textbf{4} & \textbf{5} & \textbf{6} \\ 
 \hline
 \textbf{Item Name} & VEH ID & CENTRAL BODY & && &\\
 \hline
 \textbf{Input Format} & VEH & A &&&& \\
 \hline
 \textbf{Input Units} & EBCDIC & EBCDIC&&&& \\
 \hline
 \textbf{Checking Option}&Exact (1)&Exact(2)&&&&\\
 \hline
 \textbf{Missing Item Option}&Error&Assume "E"&&&&\\
 \hline
\end{tabular}
\end{center}

\begin{tabbing}
\textbf{Notes}: \= (1) CSM or LEM\\
\> (2) \= E for Earth (=1)\\
\> \> M for moon (=3)\\
\> (3) EBCDIC name and numeric code\\
\end{tabbing}
\newpage

\textbf{MED Code}: U01\\
\textbf{Purpose}: Space digitals\\
\textbf{Example}: \texttt{U01,1,GET,100:00;00;}

\begin{center}
\begin{tabular}{|c|*{6}{>{\centering\arraybackslash}m{2.1cm}|} }
 \hline
 \diagbox{\textbf{Desc.}}{\textbf{Item}} & \textbf{1} & \textbf{2} & \textbf{3} & \textbf{4} & \textbf{5} & \textbf{6} \\ 
 \hline
 \textbf{Item Name} & MANUAL COLUMN & OPTION IND & PARAM- ETER & INCLINA- TION & ASCENDING NODE &\\
 \hline
 \textbf{Input Format} & N & AAA &HHH:MM:SS OR NN& NNN.NNN & NNN.NNN & \\
 \hline
 \textbf{Input Units} & FXP & EBCDIC&hours or FXP&deg&deg& \\
 \hline
 \textbf{Checking Option}&MINMAX(1)&Exact(2)&None&MINMAX(4)&None&\\
 \hline
 \textbf{Missing Item Option}&Error&Error&Error&None(3)&None(3)&\\
 \hline
\end{tabular}
\end{center}

\begin{tabbing}
\textbf{Notes}: \= (1) 1 $^{\leq}$ N $^{\leq}$ 3\\
\> (2) GET or MNV\\
\> (3) Mandatory when manual column = 2, otherwise illegal\\
\> (4) 0$^{\circ}$ to 180$^{\circ}$\\
\end{tabbing}
\newpage

\textbf{MED Code}: U02\\
\textbf{Purpose}: Initiate checkout monitor\\
\textbf{Example}: \texttt{U02,CSM,GET,100:00:00,90:00:00,ECT,FT;}

\begin{center}
\begin{tabular}{|c|*{6}{>{\centering\arraybackslash}m{2.1cm}|} }
 \hline
 \diagbox{\textbf{Desc.}}{\textbf{Item}} & \textbf{1} & \textbf{2} & \textbf{3} & \textbf{4} & \textbf{5} & \textbf{6} \\ 
 \hline
 \textbf{Item Name} & Veh Id & Option Ind & Parameter & Threshold Time & Reference & Feet\\
 \hline
 \textbf{Input Format} & Veh & AAA &HHH:MM:SS OR NN& HHH:MM:SS & AAA & AA\\
 \hline
 \textbf{Input Units} & EBCDIC & EBCDIC&Hours, FXP or FLP&Hours&EBCDIC&EBCDIC \\
 \hline
 \textbf{Checking Option}&Exact(2)&Exact(1)&Special(3)&None&Exact(5)&Exact(7)\\
 \hline
 \textbf{Missing Item Option}&Error&Error&Error&Special(4)&Insert(6)&Insert zero\\
 \hline
\end{tabular}
\end{center}

\textbf{Notes}:\\
(1) \begin{tabular}{c c c}
\underline{IND} & \underline{IND CODE} & \underline{PARAMETER}\\
GMT&1&TIME\\
GET&2&TIME\\
MVI&3&FXP MNV. NO.\\
MVE&4&FXP MNV. NO.\\
RAD&5&FLP RADIAL CUTOFF\\
ALT&6&FLP ALTITUDE CUTOFF\\
FPA&7&FLP FLIGHTPATH ANGLE CUTOFF\\
\end{tabular}

(2) CSM or LEM\\
(3) Parameter must be consistent with option ind.\\
(4) Optional for GET, GMT, illegal for MVI, MVE, mandatory for RAD, ALT, FPA\\
(5) ECI=0, ECT=1, MCI=2, MCT=3\\
(6) Assume ECI(=0)\\
(7) FT\\
(8) Reference indicator set negative if FT input\\
\newpage

\textbf{MED Code}: U07\\
\textbf{Purpose}: Moonrise/Moonset Display\\
\textbf{Example}: \texttt{U07,GET,100:00:00;}

\begin{center}
\begin{tabular}{|c|*{6}{>{\centering\arraybackslash}m{2.1cm}|} }
 \hline
 \diagbox{\textbf{Desc.}}{\textbf{Item}} & \textbf{1} & \textbf{2} & \textbf{3} & \textbf{4} & \textbf{5} & \textbf{6} \\ 
 \hline
 \textbf{Item Name} & IND & PARAM &&&&\\
 \hline
 \textbf{Input Format} & AAA & HHH:MM:SS or NN&&&& \\
 \hline
 \textbf{Input Units} & Note 2 &hours or FXP&&&& \\
 \hline
 \textbf{Checking Option}&Exact(1)&none or MINMAX(4)&&&&\\
 \hline
 \textbf{Missing Item Option}&Error&Error(3)&&&&\\
 \hline
\end{tabular}
\end{center}

\begin{tabbing}
\textbf{Notes}: \= (1) GET if time to be input, REV if REV to be input\\
\> (2) GET or REV\\
\> (3) Insert zero\\
\> (4) Current REV $^{\leq}$ REV $^{\leq}$ Last REV in Cape Table\\
\end{tabbing}
\newpage

\textbf{MED Code}: U08\\
\textbf{Purpose}: Sunrise/Sunset Display\\
\textbf{Example}: \texttt{U08,GET,100:00:00;}

\begin{center}
\begin{tabular}{|c|*{6}{>{\centering\arraybackslash}m{2.1cm}|} }
 \hline
 \diagbox{\textbf{Desc.}}{\textbf{Item}} & \textbf{1} & \textbf{2} & \textbf{3} & \textbf{4} & \textbf{5} & \textbf{6} \\ 
 \hline
 \textbf{Item Name} & IND & PARAM &&&&\\
 \hline
 \textbf{Input Format} & AAA & HHH:MM:SS or NN&&&& \\
 \hline
 \textbf{Input Units} & Note 2 &hours or FXP&&&& \\
 \hline
 \textbf{Checking Option}&Exact(1)&none or MINMAX(4)&&&&\\
 \hline
 \textbf{Missing Item Option}&Error&Error(3)&&&&\\
 \hline
\end{tabular}
\end{center}

\begin{tabbing}
\textbf{Notes}: \= (1) GET if time to be input, REV if REV to be input\\
\> (2) GET or REV\\
\> (3) Insert zero\\
\> (4) Current REV $^{\leq}$ REV $^{\leq}$ Last REV in Cape Table\\
\end{tabbing}
\newpage

\textbf{MED Code}: U12\\
\textbf{Purpose}: Predict apogee/perigee (FDO Orbit Digitals)\\
\textbf{Example}: \texttt{U12,CSM,GET,100:00:00;}

\begin{center}
\begin{tabular}{|c|*{6}{>{\centering\arraybackslash}m{2.1cm}|} }
 \hline
 \diagbox{\textbf{Desc.}}{\textbf{Item}} & \textbf{1} & \textbf{2} & \textbf{3a} & \textbf{3b} & \textbf{3c} & \textbf{4} \\ 
 \hline
 \textbf{Item Name} & VEH ID & IND &REV NO&TIME&MNV NO&CENTRAL BODY\\
 \hline
 \textbf{Input Format} & VEH & AAA&N&HHH:MM:SS&NN&A \\
 \hline
 \textbf{Input Units} & EBCDIC & EBCDIC&FXP&GMT&FXP&EBCDIC \\
 \hline
 \textbf{Checking Option}&Exact(3)&Exact(1)&MINMAX(2)&None&None&Exact(4)\\
 \hline
 \textbf{Missing Item Option}&Error&Error&Error&Error&Error&Assume "E"(5)\\
 \hline
\end{tabular}
\end{center}

\begin{tabbing}
\textbf{Notes}: \= (1) IND \= = REV, parameter 3a entered\\
\> \> = GET, parameter 3b entered\\
\> \> = MNV, parameter 3c entered\\
\> (2) Current REV $^{\leq}$ REV $^{\leq}$ Last REV in Cape Table\\
\> (3) CSM or LEM\\
\> (4) E for Earth (=0)\\
\> (5) This parameter is valid only when IND=REV\\
\end{tabbing}
\newpage

\textbf{MED Code}: U13\\
\textbf{Purpose}: Longitude crossing times (FDO Orbit Digitals)\\
\textbf{Example}: \texttt{U13,CSM,1,90;}

\begin{center}
\begin{tabular}{|c|*{6}{>{\centering\arraybackslash}m{2.1cm}|} }
 \hline
 \diagbox{\textbf{Desc.}}{\textbf{Item}} & \textbf{1} & \textbf{2} & \textbf{3} & \textbf{4} & \textbf{5} & \textbf{6} \\ 
 \hline
 \textbf{Item Name} & VEH ID & REV NO &Longitude&CENTRAL BODY&&\\
 \hline
 \textbf{Input Format} & VEH & NN&+DDD.XXXX&A&& \\
 \hline
 \textbf{Input Units} &EBCDIC&FXP&LONG&EBCDIC&& \\
 \hline
 \textbf{Checking Option}&Exact(2)&MINMAX(1)&None&Exact(3)&&\\
 \hline
 \textbf{Missing Item Option}&Error&Error&Error&Assume "E"&&\\
 \hline
\end{tabular}
\end{center}

\begin{tabbing}
\textbf{Notes}: \= (1) Current REV $^{\leq}$ REV $^{\leq}$ Last REV in Cape Table\\
\> (2) CSM or LEM\\
\> (3) E for Earth (=0)\\
\end{tabbing}
\newpage

\textbf{MED Code}: U14\\
\textbf{Purpose}: Compute longitude at given time (FDO Orbit Digitals)\\
\textbf{Example}: \texttt{U14,CSM,100:00:00;}

\begin{center}
\begin{tabular}{|c|*{6}{>{\centering\arraybackslash}m{2.1cm}|} }
 \hline
 \diagbox{\textbf{Desc.}}{\textbf{Item}} & \textbf{1} & \textbf{2} & \textbf{3} & \textbf{4} & \textbf{5} & \textbf{6} \\ 
 \hline
 \textbf{Item Name} & VEH ID & TIME&&&&\\
 \hline
 \textbf{Input Format} & VEH & HHH:MM:SS&&&& \\
 \hline
 \textbf{Input Units} &EBCDIC&GET&&&& \\
 \hline
 \textbf{Checking Option}&Exact(1)&None&&&&\\
 \hline
 \textbf{Missing Item Option}&Error&Error&&&&\\
 \hline
\end{tabular}
\end{center}

\begin{tabbing}
\textbf{Notes}: \= (1) CSM or LEM\\
\end{tabbing}
\newpage

\textbf{MED Code}: U20\\
\textbf{Purpose}: Generate detailed maneuver table display\\
\textbf{Example}: \texttt{U20,CSM,1;}

\begin{center}
\begin{tabular}{|c|*{6}{>{\centering\arraybackslash}m{2.1cm}|} }
 \hline
 \diagbox{\textbf{Desc.}}{\textbf{Item}} & \textbf{1} & \textbf{2} & \textbf{3} & \textbf{4} & \textbf{5} & \textbf{6} \\ 
 \hline
 \textbf{Item Name} &MPT ID&Maneuver Number&MSK Number&REFSMMAT&Heads up/ Heads down&\\
 \hline
 \textbf{Input Format} & AAA & NN&NN&AAA&A& \\
 \hline
 \textbf{Input Units} &EBCDIC&FXP&FXP&EBCDIC&D,U& \\
 \hline
 \textbf{Checking Option}&Exact(1)&MINMAX(2)&Exact(3)&Exact(4)&Note (5)&\\
 \hline
 \textbf{Missing Item Option}&Error&Error&assume 54&assume CUR&Note (6)&\\
 \hline
\end{tabular}
\end{center}

\begin{tabbing}
\textbf{Notes}: \= (1) CSM or LEM\\
\> (2) 1 $^{\leq}$ NN $^{\leq}$ 15\\
\> (3) 54 or 69\\
\> (4) CUR = 1, PCR = 2, TLM = 3, OST = 4, MED = 5, DMT = 6, DOD = 7, LCV = 8, DES = 9, LLA = 10, LLD = 11\\
\> (5) If item 4 $^{\neq}$ DES, item 5 must not be input\\
\> (6) Assume U if item 4 = DES and item 5 is missing\\
\end{tabbing}
\newpage
\end{landscape}

\section{MOCR Displays}

\subsection{FDO Launch Analog No. 1 (MSK 0040)}

The purpose of the FDO Launch Analog No. 1 is to serve as the primary display for trajectory evaluation during launch. The inertial flight-path angle ($\gamma$) in degrees versus the inertial velocity (V) in feet per second is plotted on a half-second cylce.\\

\subsection{FDO Launch Analog No. 2 (MSK 0041)}

The purpose of the FDO Launch Analog No. 2 is to display conditions of ($\gamma$, V)\textsubscript{EI}. The flight-path angle versus inertial velocity at entry interface ($\gamma$ vs V)\textsubscript{EI} is plotted. The plot is initialized when apogee altitude is equal to or above entry interface. It is then updated and terminated at reaching orbit.\\

\subsection{FDO Launch Digital No. 1 (MSK 0043)}

TBD.

\subsection{FDO Orbit Digitals (MSK 0045 and 0046)}

\subsubsection{Function}

The function of the Flight Dynamics Officer (FDO) Orbit Digitals Display is to compute present position information and predicted data concerning apogee/poergee and longitude crossings as manually requested.\\
\newpage
\subsubsection{Display Parameters}

\begin{center}
\begin{tabular}{ L{3cm} L{4.5cm} L{3cm} L{3.8cm} }
 \underline{QUANTITY} & \underline{DEFINITION} & \underline{DIMENSIONS} & \underline{UPDATES} \\ 
Vehicle & Vehicle name & & \\  
Rev & Current revolution number associated with subject vehicle and central body & & (1)\\
Vector ID & Last vector used for updating the ephemeris && (2)\\
GMT ID & GMT Time of anchor vector & Hr:min:sec & (2)\\
GET ID & GET Time of anchor vector & Hr:min:sec & (2)\\ 
GET & Current ground elapsed time & Hr:min:sec & (1)\\
Weight & Total current weight of subject vehicle & lbs. &(1)\\
K-Factor & Atmospheric density multiplier considered in generating the ephemeris&&(1), (2)\\
$\lambda$\textsubscript{PP}&Present position, longitude (Seleno- graphic for moon)& deg. & (1), (2)\\
$\phi$\textsubscript{PP}&Present position, latitude (geodetic for earth, Selenographic for moon)& deg. & (1), (2)\\
GET\textsubscript{CC}* & GET of arrival at next cape crossing&Hr:min:sec&(1), (2)\\
\end{tabular}
\end{center}

\line(1,0){200}
(1) Output cycle (12 sec.)\\
(2) Trajectory Update, MSK Request\\
$\ast$ For lunar orbit this will be the time of the vehicle crossing of the 180$\deg$ selenographic longitude.\\
\newpage
\begin{center}
\begin{tabular}{ L{3cm} L{4.5cm} L{3cm} L{3.8cm} }
 \underline{QUANTITY} & \underline{DEFINITION} & \underline{DIMENSIONS} & \underline{UPDATES} \\ 
N\textsubscript{PP}&True anomaly&deg.&(1), (2)\\
$\lambda$\textsubscript{AN}&Longitude of the ascending node (earth fixed or moon fixed)&deg.&Node crossing, (2)\\
h&Current oblate height (Spherical for moon)&N.M.&(1), (2)\\
V\textsubscript{i}&Current inertial velocity&f.p.s.&(1), (2)\\
$\gamma$&Current inertial flightpath angle&deg.&(1), (2)\\
a&Semimajor axis of orbital ellipse&N.M.&(1), (2)\\
e&Eccentricity of orbital ellispe&&(1), (2)\\
i&Orbital inclination to central body equator&deg.&(1), (2)\\
GET\textsubscript{EI}&GET of arrival at unsafe altitude&Hr:min:sec&(2)\\
$\phi$\textsubscript{EI}&Geodetic latitude of entry interface&deg.&(2)\\
$\lambda$\textsubscript{EI}&Longitude of entry interface&deg.&(2)\\
PET&Phase elapsed time (GET of an event minus GETR)&Hr:min:sec&(4)\\
\end{tabular}
\end{center}

\line(1,0){200}
(1) Output cycle (12 sec.)\\
(2) Trajectory Update, MSK Request\\
(4) These times will be updated when a new GETR and a new event is specified.\\
\newpage

\begin{center}
\begin{tabular}{ L{2.9cm} L{4.6cm} L{3cm} L{3.8cm} }
\underline{QUANTITY} & \underline{DEFINITION} & \underline{DIMENSIONS} & \underline{UPDATES} \\
GETR&GET of reference (elapsed time of an event)&Hr:min:sec&(4)\\
GET\textsubscript{$\lambda$}&Time S/C will pass over $\lambda$ below&Hr:min:sec&Computed or manually entered\\
REV\textsubscript{$\lambda$}&Revolution associated with GET\textsubscript{$\lambda$}&&Manually entered or computed\\
$\lambda$&The longitude associated with above GET&deg.&Manually entered or computed\\
h\textsubscript{a}$\ast$$\ast$&Spherical height of next apogee at GET\textsubscript{a}&N.M.&Crossing of Apogee, (2), maneuver executed\\
$\phi$\textsubscript{a}$\ast$$\ast$&Latitude of next apogee at GET\textsubscript{a}. Geocentric in earth reference, seleno- graphic around moon&deg.&Crossing of Apogee, (2), maneuver executed\\
$\lambda$\textsubscript{a}$\ast$$\ast$&Longitude of next apogee at GET\textsubscript{a}&deg.&Crossing of Apogee, (2), maneuver executed\\
GET\textsubscript{a}$\ast$$\ast$&Time of arrival at next apogee&Hr:min:sec&Crossing of Apogee, (2), maneuver executed\\
h\textsubscript{p}$\ast$$\ast$&Spherical height of next perigee at GET\textsubscript{p}&N.M.&Crossing of Perigee, (2), maneuver executed\\
$\phi$\textsubscript{p}$\ast$$\ast$&Latitude at next perigee at GET\textsubscript{p}. Geocentric relative to the earth , selenographic around moon&deg.&Crossing of Perigee, (2), maneuver executed\\
$\lambda$\textsubscript{p}$\ast$$\ast$&Longitude of next perigee at GET\textsubscript{p}&deg.&Crossing of Perigee, (2), maneuver executed\\
GET\textsubscript{p}$\ast$$\ast$&Time of arrival at next perigee&Hr:min:sec&Crossing of Ppogee, (2), maneuver executed\\
GET\textsubscript{BV}&Time tag of vector from which apogee/perigee values were computed&Hr:min:sec&Computed or manually entered\\
REV\textsubscript{BV}&Revolution of GET\textsubscript{BV}&&Computed or manually entered
\end{tabular}
\end{center}

\line(1,0){200}
(2) Trajectory Update, MSK Request\\
(4) These times will be updated when a new GETR and a new event is specified.\\
$\ast$$\ast$ Also displayed manually for future time periods as requested.\\
\newpage

\begin{center}
\begin{tabular}{ L{2.9cm} L{4.6cm} L{3cm} L{3.8cm} }
\underline{QUANTITY} & \underline{DEFINITION} & \underline{DIMENSIONS} & \underline{UPDATES} \\
T\textsubscript{0}&Orbital Period&Hr:min:sec&(1)\\
REF1, REF2, REF3&Indicator of which central body is being referenced for present position data and for manual Apogee/Perigee and longitude&&\\
NV\textsubscript{1}&Number of vectors used for interpolation for present position values&&\\
NV\textsubscript{2}&Number of vectors used in interpolation for base vector for predicted apogee/perigee data&&\\
Trajectory Update Number&Update number associated with subject vehicle ephemeris&&(2)\\
\end{tabular}
\end{center}

\subsection{Mission Plan Table (MSK 0047)}

\subsubsection{Function}

\newpage

\subsection{Space Digitals (MSK 0082)}

\subsubsection{Function}

The function of the Space Digitals Display is to compute and display parameters necessary to evaluate and monitor trajectories involving Earth-moon relationships. At the top of the display present position data are shown. In column 1 orbital elements at an input time are displayed. Column 2 has the time of lunar sphere of influence entry, numbers for the closest approach to the Moon and at the node, the intersection with the lunar orbit plane (TBD). The third column shows the time of lunar sphere of influence exit, the closest approach to Earth after that exit and, if applicable, the state at Earth entry interface.

\subsubsection{Display}

The update associated with this display are:\\

(1) Time cycle (12 sec)\\
(2) Trajectory update\\
(3) Manual entries (3)\\
(4) MSK request\\
(5) Reinitialization\\
\begin{center}
\begin{tabular}{ L{2.4cm} L{5.1cm} L{3cm} L{3.8cm} }
\underline{QUANTITY} & \underline{DEFINITION} & \underline{DIMENSIONS} & \underline{UPDATES} \\
Vector-ID & Identification of the last vector used to update the ephemeris & None &(1), (2), (4), (5)\\
Weight & Total vehicle weight & lbs & (1), (2), (4), (5)\\
GMTV & Greenwich time-tag of the vector & HHH:MM:SS.SS &(1), (2), (4), (5)\\
GETV & Ground elapsed time-tag of the vector & HHH:MM:SS.SS &(1), (2), (4), (5)\\
GET Axis & Ground elapsed time used to define the earth-moon line in the initialization of the earth-moon transit display&HHH:MM:SS.SS &(1), (2), (4), (5)\\
GETR & Ground elapsed time reference (elapsed time of an event) &HHH:MM:SS.SS &(1), (2), (4), (5)\\
GET & Current ground elapsed time for which V, GAM, PHI, LAM, H, PSI and ADA were computed&HHH:MM:SS.SS &(1)\\
REF & Inertial reference body used to compute V, GAM, PHI, LAM, H, PSI and ADA&HHH:MM:SS.SS &(1)\\
V & Current velocity (earth centered inertial or lunar centered inertial) & ft/sec &(1), (2), (4), (5)\\
\end{tabular}
\end{center}

\begin{center}
\begin{tabular}{ L{2.4cm} L{5.1cm} L{3cm} L{3.8cm} }
\underline{QUANTITY} & \underline{DEFINITION} & \underline{DIMENSIONS} & \underline{UPDATES} \\
GAM & Current inertial flightpath angle (earth or lunar reference) & deg & (1), (2), (4), (5)\\
H & Current altitude above an oblate earth or spherical above the moon assuming landing site radius & nm &(1), (2), (4), (5)\\
PHI & Current geodetic latitude or current selenographic latitude respect to a spherical moon whose radius is the landing site & deg &(1), (2), (4), (5)\\
LAM & Current earth longitude or current selenographic longitude & deg &(1), (2), (4), (5)\\
PSI & Current heading angle with respect to the earth or moon, measured clockwise from north & deg &(1), (2), (4), (5)\\
ADA & Current true anomaly & deg &(1), (2), (4), (5)\\
VEH ID & Vehicle for which the space digitals are computed & XXX & (2), (3), (4), (5)\\
GET Vector 1 & Ground elapsed time of the vector used to compute the quantities below & HHH:MM:SS.SS & (3)\\
REF & The inertial reference body used to compute the quantities (i.e., if GET Vector 1 is inside the lunar sphere of influence, REF is MOON) & None & (3)\\
WT & Total weight at time of GET Vector 1 & lbs & (3)\\
GETA & Ground elapsed time of next apogee (reference from GET Vector 1) & HHH:MM:SS.SS & (3)\\
HA & Height of apogee above a spherical earth or height of apolune above a spherical moon whose radius is the landing site (referenced from GET Vector 1) & nm & (3)\\
HP & Height of perigee above a spherical earth or height of perilune above a spherical moon whose radius is the landing site (referenced from GET Vector 1) & nm & (3)\\
\end{tabular}
\end{center}

\begin{center}
\begin{tabular}{ L{2.4cm} L{5.1cm} L{3cm} L{3.8cm} }
\underline{QUANTITY} & \underline{DEFINITION} & \underline{DIMENSIONS} & \underline{UPDATES} \\
IEMP & Inclination of the trajectory with respect to the earth-moon plane (referenced from GET Vector 1) & deg & (3)\\
HS & Altitude above a spherical earth or moon (referenced from GET Vector 1) & nm & (3)\\
PHI & Geocentric latitude or selenographic latitude with respect to a spherical moon of landing site radius (referenced from GET Vector 1) & deg & (3)\\
OMG & Right ascension of the ascending node (inertial) & deg & (3)\\
PRA & Inertial right ascension of perigee & deg:min & (3)\\
K & Density multiplier used in the computation of the above quantities & None & (3)\\
GET Vector 2 & Ground elapsed time of the vector supplied by the ephemeris and from which the quantities below will be computed & HHH:MM:SS.SS & (3)\\
GETSI Sphere Entrance & The first ground elapsed time after GET Vector 2 that the vehicle will pass through the lunar sphere of influence with a negative lunar altitude rate (rate with respect to the moon is decreasing) & HHH:MM:SS.SS & (3)\\
GETCA & Ground elapsed time of closest approach to the moon & HHH:MM:SS.SS & (3)\\
VCA &  Velocity with respect to a moon-centered inertial coordinate system at the point of closest approach to the moon & ft/sec & (3)\\
HCA & Altitude at the point of closest approach to a spherical moon of landing site radius & nm & (3)\\
PSICA & Heading angle at the point of closest approach to the moon measured clockwise from north & deg & (3)\\
PHICA & Selenographic latitude with respect to a spherical moon at the point of closest approach to the moon & deg & (3)\\
\end{tabular}
\end{center}

\begin{center}
\begin{tabular}{ L{2.4cm} L{5.1cm} L{3cm} L{3.8cm} }
\underline{QUANTITY} & \underline{DEFINITION} & \underline{DIMENSIONS} & \underline{UPDATES} \\
LCA & Selenographic longitude with respect to the moon & deg & (3)\\
GETMN & Ground elapsed time of arrival at the node defined by the planes of the approach hyperbola and the desired lunar parking orbit & HHH:MM:SS.SS & (3)\\
PMN & Selenographic latitude relative to a spherical moon at GETMN & deg & (3)\\
LMN & Selenographic longitude relative to a spherical moon at GETMN & deg & (3)\\
HMN & Selenographic height relative to a spherical moon of landing site radius & nm & (3)\\
DMN & Wedge angle between the planes of the approach hyperbola and the desired lunar parking orbit (Note: A positive sign indicates that the inclination of the desired lunar parking orbit is less than the inclination of the approach hyperbola) & deg & (3)\\
GET Vector 3 & Ground elapsed time of the vector supplied by the ephemeris and from which the quantities below will be computed & HHH:MM:SS.SS & (3), (2)\\
GETSE Sphere Exit & The first ground elapsed time after GET Vector 3 that the vehicle will pass through the lunar sphere of influence with a positive lunar altitude rate (Note: Altitude with respect to the moon is increasing) & HHH:MM:SS.SS & (3), (2)\\
GETEI & Ground elapsed time of entry interface & HHH:MM:SS.SS & (3), (2)\\
VEI & Earth centered velocity at entry interface & ft/sec & (3), (2)\\
GEI & Inertial flightpath angle at entry interface & deg & (3), (2)\\
PEI & Geodetic latitude at entry interface & deg & (3), (2)\\
LEI & Earth longitude at entry interface & deg & (3), (2)\\
\end{tabular}
\end{center}

\begin{center}
\begin{tabular}{ L{2.4cm} L{5.1cm} L{3cm} L{3.8cm} }
\underline{QUANTITY} & \underline{DEFINITION} & \underline{DIMENSIONS} & \underline{UPDATES} \\
PSIEI & Heading angle with respect to earth at entry interface & deg & (3), (2)\\
GETVP & Ground elapsed time of arrival at vacuum perigee & HHH:MM:SS.SS & (3)\\
VVP & Earth centered velocity at vacuum perigee & ft/sec & (3)\\
HVP & Altitude above an oblate earth at vacuum perigee & nm & (3)\\
PVP & Geodetic latitude at vacuum perigee & deg & (3)\\
LVP & Earth longitude at vacuum perigee & deg & (3)\\
PSIVP & Heading angle with respect to earth at vacuum perigee & deg & (3)\\
IE & Inclination angle with respect to the earth at vacuum perigee & deg & (3)\\
LN & Geographic longitude of the earth return ascending node & deg & (3)\\
\end{tabular}
\end{center}

\newpage
\subsection{Vector Comparison Display (MSK 1591)}

\subsubsection{Function}

The Vector Comparison Display shows the local spherical elements, the classical elements and UVW coordinates for a base (first) vector. The differences between the elements of this vector and the second, third, and fourth vectors (if input) will then be computed. From one to four vectors can be specified for comparison. Any available state vector in the evaluation or usable vector tables can be used (see Vector Panel Summary).\\

The UVW coordinate system, essentially a local vertical, local horizontal coordinate system, is defined as follows:

\begin{itemize}
	\item X Axis (U): Vector pointing in the direction of the spacecraft's position vector\\
	\item Y Axis (V): Vector perpendicular to the x- and z-axes, pointing in the direction of travel for a circular orbit\\
	\item Z Axis (W): Vector pointing in the direction normal to the orbit (along the angular momentum vector)\\
\end{itemize}

\subsubsection{Buttons}

\textbf{TIM:} Time of comparison. GMT is assumed if input is positive, GET if negative. Time of V1 vector will be used for comparison if zero is used as the input time.\\
\textbf{VEH:} Vehicle for the comparison. Only relevant if mission planning mode is active in the MFD and an ephemeris vector was chosen.\\
\textbf{V1:} Choose ID of first (base) vector from vector panel summary.\\
\textbf{V2:} Choose ID of second state vector.\\
\textbf{V3:} Choose ID of third state vector.\\
\textbf{V4:} Choose ID of fourth state vector.\\

\textbf{CLC:} Calculate display parameters.\\
\textbf{REF:} Reference body (Earth or Moon) for the comparison values.\\
\textbf{BCK:} Back to last menu\\

\newpage
\begin{landscape}
\subsubsection{Display Parameters}

\begin{center}
\begin{tabular} { L{1cm} L{8cm} L{2cm} L{6cm} }
\hline
Name & Definition & Unit & Comment\\
\hline
H\textsubscript{a} & Keplerian height of apogee (1st column); difference in height of apogee from reference vector (2nd through 4th columns) & n.mi.&Cannot be computed for hyperbolic or parabolic orbits.\\
H\textsubscript{p} & Keplerian height of perigee (1st column); difference in height of perigee from reference vector (2nd through 4th columns) & n.mi.&Cannot be computed for parabolic orbits.\\
V&Inertial velocity magnitude (1st column); difference in velocity from reference vector (2nd through 4th columns).&ft/sec&\\
$\gamma$&Flightpath angle (1st column); difference in angle from reference vector (2nd through 4th columns).&deg&\\
$\psi$&Azimuth (1st column); difference in azimuth from reference vector (2nd through 4th columns).&deg&\\
$\phi$&Latitude (1st column); difference in latitude from reference vector (2nd through 4th columns)&deg&\\
$\lambda$&Longitude (1st column);difference in longitude from reference vector (2nd through 4th columns).&deg\\
h&Height above launch pad (Earth) or lunar landing site (1st column);difference in height from reference vector (2nd through 4th columns).&n.mi.&\\
\end{tabular}
\end{center}
\newpage
\begin{center}
\begin{tabular} { L{1cm} L{8cm} L{2cm} L{6cm} }
\hline
Name & Definition & Unit & Comment\\
\hline
a&Semimajor axis (1st column); difference in semimajor axis from reference vector (2nd through 4th columns).&n.mi.&Cannot be computed for parabolic orbit.\\
e&Eccentricity (1st column); difference in eccentricity from reference vector (2nd through 4th columns)&none&\\
i&Inclination (1st column); difference in inclination from reference vector (2nd through 4th columns)&deg&\\
$\omega$\textsubscript{p}&Argument of perigee (1st column);difference in $\omega$\textsubscript{p} from reference vector (2nd through 4th columns)&deg&\\
$\Omega$&Right ascension of the ascending node (1st column);difference in $\Omega$ from reference vector (2nd through 4th columns)&deg&\\
$\nu$&True anomaly (1st column);difference in true anomaly from reference vector (2nd through 4th columns)&deg&\\
UVW&UVW position coordinates of vehicle (1st column);difference in UVW coordinates from reference vector in feet (2nd through 4th columns).&n.mi., ft&Differences are computed in feet\\
$\dot{U}\dot{V}\dot{W}$& velocity components (1st column);difference in UVW velocity from reference vector (2nd through 4th columns).&ft/sec&\\
\end{tabular}
\end{center}

\end{landscape}

\subsection{Vector Panel Summary (MSK 1591)}

\subsubsection{Function}

The Vector Panel Summary will display the ephemeris anchor vector identifications, anchor vector times and current GMT. It will also display the vector identification and time for each vector in a Usable Vector Slot and each vector in an Evaluation Vector Slot. GMT of ullage tailoff are displayed for the last executed maneuver for both vehicles. The time tags of all available telemetry vectors will also be displayed. The display is updated every six seconds, so inputs on the page might not cause an instant change to the display.

\subsubsection{Display Parameters and Buttons}

The left and right half of the display are showing information about state vectors for CSM and LM respectively. The two sides are identical, so the following description applies to both. Each side is divided into several panels: CMC, LGC, AGS, IU, High Speed Radar, Differential Correction (DC) and Last Executed Maneuver. The first four panels show information about state vectors from telemetry and vectors that have been moved to RTCC tables:\\

UV = Usable Vector Table\\
EV = Evaluation Vector Table\\
TH = Telemetry (High Speed)\\
TL = Telemetry (Low Speed)\\

The TL vector slot has not been implemented yet. Pressing the TLM button will get data from all computers in the vessel where the RTCC MFD is open (AGS not yet implemented) and save them in the high speed telemetry tables. In reality the telemetry vectors get continually updated if telemetry is received, so for a more permanent saving solution the vectors can be moved to the evaluation vector table with the EV buttons. The vector is then moved to that table, assigned a vector ID and is then available for comparison using the vector compare display (MSK 1590). The telemetry vectors are not guaranteed to be valid, so the evaluation vectors were checked before being "moved up" to the usable vector table, which can be done with the UV buttons. Once in the usable vector table the vectors are available for ephemeris updates.\\

The High Speed Radar panel is not currently implemented. It was a vector that is generated automatically at the end of a powered maneuver using radar data and could, if no telemetry vector was available, be a first estimate of the actual cutoff state.\\

Differential Correction (DC) is the state vector derived from long period ground tracking of the spacecraft during coasting flight. In the RTCC MFD it is simply generated with the DC button and entering the name of the vessel in the simulation. The DC panel will then be updated with a vector ID (APIC/APIL for Orbiter API and C/L for the CSM or LM) and a vector time. The HSR and DC vectors are moved directly to the usable vector table.\\

The last panel shows the last executed maneuver in the mission plan table of the vehicle. The displayed times are ullage on and end of the thrust tailoff period after cutoff.\\

To cause an ephemeris update the TUP button on each side can be pressed and then a vector from the usable vector table is chosen to do the update.\\

\newpage
\section{Config Files}

The RTCC loads various configuration files to load constants, mission and launch day specific parameters. The files are located at Config\textbackslash ProjectApollo\textbackslash RTCC.

\subsection{Star Table}

The RTCC star table is stored in the Star Table.txt file. It contains unit vector of all RTCC navigation stars, starting with the star that get used by the Apollo Guidance Computer. The file is equivalent to the RTCC mission table named EZJGSTAR.

\subsection{Mission Constants}

For each Apollo mission a file with mission constants (e.g. "Apollo 8 Constants.txt") is loaded. These numbers are constant for all possible launch days of a mission. They usually are related to hardware or software which doesn't get changed for a different launch day. Some of these parameters are saved and loaded in scenarios.\\

\begin{tabular}{L{3cm} L{7.5cm} L{2cm} L{1.5cm}}
\hline
\textbf{Name} & \textbf{Description} & \textbf{Default Value} & \textbf{Unit}\\
\hline
AGCEpoch & The epoch of the coordinate system used by the Apollo Guidance Computer (and in the future the RTCC). Used to convert state vectors to the uplink coordinate system. & 40221.525 (Apollo 7-10) & Days (MJD)\\
\hline
MCCLEX & Octal LGC address for external DV uplink & 3433 & None\\
\hline
MCCLRF & Octal LGC address for REFSMMAT uplink (and downlink)& 1733 & None\\
\hline
MCCCXS & Octal CMC address for desired REFSMMAT uplink & 306 & None\\
\hline
MCCLXS & Octal LGC address for desired REFSMMAT uplink & 3606 & None\\
\hline
MCCCRF & Octal CMC address for REFSMMAT uplink (and downlink) & 1735 & None\\
\hline
MCCLRF & Octal LGC address for REFSMMAT uplink (and downlink) & 1733 & None\\
\hline
MCLRLS & LGC address for landing site vector uplink (and downlink) & 2022 & None\\
\hline
PZREAP\_RRBIAS & Relative range override for the Return-to-Earth processor & 1285.0 & Nautical Miles\\
\hline
PZREAP\_IRMAX & Maximum return inclination for the Return-to-Earth processor & 40.0 & Degrees\\
\hline
MCLADA & Geodetic latitude of the launch pad used for launch REFSMMAT calculation & 28.608202 (LC-39A) & Degrees\\
\hline
MDVSTP\_PHIL & Geodetic latitude of the launch pad used for TLI simulation and LVDC state vector uplink calculation & 28.608202 (LC-39A) & Degrees\\
\hline
MCLGRA & Longitude of the launch pad & -80.604133 (LC-39A) & Degrees\\
\hline
\end{tabular}

\newpage
\begin{tabular}{L{3cm} L{7cm} L{2.5cm} L{1.5cm}}
\hline
\textbf{Name} & \textbf{Description} & \textbf{Default Value} & \textbf{Unit}\\
\hline
PDI\_v\_IGG & PDI ignition algorithm velocity & 5545.46 & Feet per second\\
\hline
PDI\_r\_IGXG & PDI ignition algorithm x-axis position & -130519.86 & Feet\\
\hline
PDI\_r\_IGZG & PDI ignition algorithm z-axis position & -1432597.3 & Feet\\
\hline
PDI\_K\_X & PDI ignition algorithm coefficient & 0.617631 & None\\
\hline
PDI\_K\_Y & PDI ignition algorithm coefficient & 0.755e-6 & Feet per Feet$^2$\\
\hline
PDI\_K\_V & PDI ignition algorithm coefficient & 410.0 & Seconds\\
\hline
PDI\_RBRFG & PDI braking phase position target & 171.835, 0.0, -10678.596 & Feet\\
\hline
PDI\_VBRFG & PDI braking phase velocity target & -105.876, 0.0, -1.04 & Feet per second\\
\hline
PDI\_ABRFG & PDI braking phase acceleration target & 0.6241, 0.0, -9.1044 & Feet per second$^2$\\
\hline
PDI\_JBRFGZ & PDI braking phase jerk & -0.01882677 & Feet per second$^3$\\
\hline
PDI\_RARFG & PDI approach phase position target & 111.085, 0.0, -26.794 & Feet\\
\hline
PDI\_VARFG & PDI approach phase velocity target & -4.993, 0.0, 0.248 & Feet per second\\
\hline
PDI\_AARFG & PDI approach phase acceleration target &  -0.2624, 0.0, -0.512 & Feet per second$^2$\\
\hline
PDI\_JARFGZ & PDI approach phase jerk & 0.00180772 & Feet per second$^3$\\
\hline
\end{tabular}

\newpage
\subsection{Launch Day Init Parameters}

For each launch day of a given mission (e.g. "1968-12-21 Init.txt" for Apollo 8) a few parameters are being loaded as initial values. These parameters are almost all saved and loaded and may be overwritten during a mission, so they are only loaded from file once when the RTCC is first being started.\\

\begin{tabular}{L{3.7cm} L{7cm} L{1.95cm} L{1.5cm}}
\hline
\textbf{Name} & \textbf{Description} & \textbf{Default Value} & \textbf{Unit}\\
\hline
LSLat & Latitude of the lunar landing site & 0.0 & Degrees\\
\hline
LSLng & Longitude of the lunar landing site & 0.0 & Degrees\\
\hline
LSRad & Radius of the lunar landing site & Mean lunar radius & Nautical miles\\
\hline
LDPPDwellOrbits & Number of dwell orbits desired between DOI and PDI used by the Lunar Descent Planning Processor & 0 & None\\
\hline
LDPPDescentFlightArc & Powered flight arc of descent for the Lunar Descent Planning Processor & 15.0 & Degrees\\
\hline 
LDPPHeightofPDI & Height of PDI used by the Lunar Descent Planning Processor & 50000.0 & Feet\\
\hline
PZLOIPLN\_HP\_LLS & Height of perilune at PDI for the LOI Targeting & 8.23 & Nautical Miles\\
\hline
TLCC\_AZ\_min & Minimum approach azimuth to the landing site used by the Translunar Midcourse Processor & -110.0 & Degrees\\
\hline
TLCC\_AZ\_max & Maximum approach azimuth to the landing site used by the Translunar Midcourse Processor & -70.0 & Degrees\\
\hline
REVS1 & Number of orbits between LOI-1 and LOI-2 (or DOI for the later missions) used by the TLMCC and LOI targeting & 2.0 & None (double)\\
\hline
REVS2 & Number of orbits between LOI-2 (or DOI for the later missions) and PDI used by the TLMCC and LOI targeting & 11 & None (Integer)\\
\hline
LOPC\_M & Number of revolutions from first pass over lunar landing site to lunar orbit plane change maneuver. Used by LOPC routine of the TLMCC processor & 3 & None\\
\hline
LOPC\_N & Number of revolutions from lunar orbit plane change maneuver to second pass over lunar landing site. Used by LOPC routine of the TLMCC processor & 8 & None\\
\hline
\end{tabular}

\begin{tabular}{L{3.7cm} L{7cm} L{1.95cm} L{1.5cm}}
\hline
\textbf{Name} & \textbf{Description} & \textbf{Default Value} & \textbf{Unit}\\
\hline
LOI\_psi\_DS & Desired approach azimuth to the lunar landing site. Used by the LOI targeting & 270.0 & Degrees\\
\hline
eta\_1 & True anomaly on LPO-1 (orbit after LOI-1) for transferring from the hyperbola to LPO-1. Used by TLMCC and LOI targeting & 0.0 & Degrees\\
\hline
H\_P\_LPO1 & Perilune height on LPO-1 (orbit after LOI-1). Used by TLMCC and LOI targeting & 60.0 & Nautical miles\\
\hline
PZLTRT\_DT\_Ins\_TPI & Time between insertion and TPI for the short rendezvous profile in lunar orbit & 40.0 & Minutes\\
\hline
SITEROT & Angle of perilune from the lunar landing site (negative if the site is post-perilune). Used by the TLMCC and LOI targeting & -15.0 & Degrees\\
\end{tabular}

\newpage
\subsection{Skeleton Flight Plan Table}

The skeleton flight plan table is a table of initial guesses for the translunar midcourse correction processor. The files are launch day specific and have the format e.g. "1968-12-21 SFP.txt".\\

\begin{tabular}{L{3.7cm} L{7cm} L{2.5cm}}
\hline
\textbf{Name} & \textbf{Description} & \textbf{Unit}\\
\hline
SFP\_DPSI\_LOI & Delta Azimuth of LOI & Degrees\\
\hline
SFP\_DPSI\_TEI & Delta Azimuth of TEI & Degrees\\
\hline
SFP\_DT\_LLS & Time from LOI to the first pass (time of landing) over the landing site & Hours\\
\hline
SFP\_DV\_TEI & Delta velocity of the TEI maneuver & Feet per second\\
\hline
SFP\_GAMMA\_LOI & Flight path angle before LOI maneuver & Degrees\\
\hline
SFP\_GET\_TLI & Ground elapsed time of TLI ignition & Hours\\
\hline
SFP\_GMT\_ND & GMT at the node & Hours\\
\hline
SFP\_GMT\_PC1 & GMT of TLI pericynthion & Hours\\
\hline
SFP\_GMT\_PC2 & GMT of LOI pericynthion & Hours\\
\hline
SFP\_H\_ND & Height of the node & Nautical miles\\
\hline
SFP\_H\_PC1 & Height of TLI pericynthion & Nautical miles\\
\hline
SFP\_H\_PC2 & Height of LOI pericynthion & Nautical miles\\
\hline
SFP\_INCL\_FR & Inclination of free return & Degrees\\
\hline
SFP\_LAT\_LLS & Latitude of the lunar landing site & Degrees\\
\hline
SFP\_LAT\_ND & Latitude of the node & Degrees\\
\hline
SFP\_LAT\_PC1 & Latitude of TLI pericynthion & Degrees\\
\hline
SFP\_LAT\_PC2 & Latitude of LOI pericynthion & Degrees\\
\hline
SFP\_LNG\_LLS & Longitude of the lunar landing site & Degrees\\
\hline
SFP\_LNG\_ND & Longitude of the node & Degrees\\
\hline
SFP\_LNG\_PC1 & Longitude of TLI pericynthion & Degrees\\
\hline
SFP\_LNG\_PC2 & Longitude of LOI pericynthion & Degrees\\
\hline
SFP\_PSI\_LLS & Approach azimuth to the landing site & Degrees\\
\hline
SFP\_RAD\_LLS & Radius of the lunar landing site & Nautical miles\\
\hline
SFP\_T\_LO & Time in lunar orbit (LOI to TEI) & Hours\\
\hline
SFP\_T\_TE & Time of transearth coast (TEI to entry interface) & Hours\\
\hline
\end{tabular}
 
\newpage
\subsection{TLI Targeting Parameters}

The files containing the TLI targeting parameters follow the launch day specific format "`1968-12-21 TLI.txt". They contain the numbers needed to calculate the time of restart preparations and the TLI maneuver in the RTCC. The format of the file follows the MSC card format from the MSC internal note RTCC REQUIREMENTS FOR APOLLO 11 (MISSION G) PREFLIGHT INFORMATION (69-FM-171), which can be found \href{https://web.archive.org/web/20100524010957/http://ntrs.nasa.gov/archive/nasa/casi.ntrs.nasa.gov/19740072570_1974072570.pdf}{here}. The PDF pages 32 to 40 show the format. One line in the config file is equal to one punch card. Note that only the first day of a monthly launch window is stored per file, instead of the up to ten.

\end{document}
